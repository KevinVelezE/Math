\documentclass{amsart}
% Language and encoding
\usepackage[spanish]{babel}
\usepackage[utf8]{inputenc}

% margins
% \usepackage[lmargin=2.5cm, rmargin=2cm]{geometry}
\usepackage[lmargin=3cm, rmargin=2cm, bmargin=2cm, tmargin=2cm]{geometry}

% Math packages
\usepackage{amsmath}
\usepackage{amssymb}
\usepackage{amsthm}

% graphics and color
\usepackage{graphicx}
\usepackage{xcolor}
\usepackage{float}
\usepackage{subfigure}
\usepackage{wrapfig}

% color boxes 
\usepackage[most,many,breakable]{tcolorbox}

% fancy headers
\usepackage{fancyhdr}

% Tikz
\usepackage{tikz}
\usetikzlibrary{angles, calc, arrows, arrows.meta}
\AtBeginEnvironment{tikzpicture}{\shorthandoff{>}\shorthandoff{<}}{}{}

% cancel
\usepackage[makeroom]{cancel}
\newcommand\cancelc[2][black]{\renewcommand\CancelColor{\color{#1}}\cancel{#2}}

% fancy settings
\setlength{\headheight}{18pt}
\pagestyle{fancy}
\fancyhf{}
\fancyhead[L]{\includegraphics[height=5mm]{./figures/logo}}
\fancyfoot[R]{\thepage}

% date format
\renewcommand{\datename}{\emph{Fecha:}}


% ------------------ %

% Theorem environments

% colors for theorem environments
\definecolor{myqsbg}{HTML}{f2fbfc}
\definecolor{myqsfr}{HTML}{191971}

% problem environment
\tcbuselibrary{theorems,skins,hooks}
\newtcbtheorem{question}{Problema}
{
	enhanced,
	breakable,
	colback = myqsbg,
	frame hidden,
	boxrule = 0sp,
	borderline west = {2pt}{0pt}{myqsfr},
	sharp corners,
	detach title,
	before upper = \tcbtitle\par\smallskip,
	coltitle = myqsfr,
	fonttitle = \bfseries\sffamily,
	description font = \mdseries,
	separator sign none,
	segmentation style={solid, myqsfr},
}
{th}

% Environment for problems
\newtheorem{problem}{Problema}

% simplified theorem environment

\newcommand{\qs}[2]{\begin{question}{#1}{}#2\end{question}}

% solution command
\newcommand{\sol}[1]{\noindent\textbf{\textit{Solución:}} #1 \par \hfill $_\square$ }
\newcommand{\dem}[1]{\noindent\textbf{\textit{Demostración:}} #1 \par \hfill $_\square$ }

% shortcuts for some math symbols
\newcommand{\R}{\mathbb{R}}
\newcommand{\N}{\mathbb{N}}
\newcommand{\Z}{\mathbb{Z}}
\newcommand{\Q}{\mathbb{Q}}
\newcommand{\C}{\mathbb{C}}

\newcommand{\Bb}{\mathcal{B}}
\newcommand{\ps}[1]{\mathcal{P}\left(#1\right)}

\newcommand{\dd}{\mathrm{d}}
\newcommand{\ZnZ}{\Z / n\Z}

% shortcuts for some math operators
\newcommand{\abs}[1]{\left\lvert #1 \right\rvert}
\newcommand{\norm}[1]{\left\lVert #1 \right\rVert}

\newcommand{\ol}[1]{\overline{#1}}

\title{Taller 1 - Lógica Matemática}
\author{Kevin Velez Escarria}
\date{Octubre, 2022}

\begin{document}
\maketitle \thispagestyle{fancy}

\qs{}{
    Mostrar que si de $\Gamma \vdash \alpha $ y de $ \alpha \vdash \neg \beta$ entonces $\Gamma \vdash \beta$
}
\dem{}

\qs{}{
    Demostrar que:
    \begin{enumerate}
        \item[a)] $\vdash (\neg \neg \beta \to \neg \alpha) \to (\alpha \to \neg \beta)$
        \item[b)] $\alpha \to \beta, \beta \to \gamma \vdash \alpha \to \gamma$
        \item[c)] $(\alpha \to \beta) \vdash \neg \beta \to \neg \alpha$
        \item[d)] $ \vdash (\beta \to \gamma) \to ((\alpha\to\beta) \to (\alpha\to\gamma)) $ 
        \item[e)] $\vdash (\alpha \to \beta) \to ((\beta\to\gamma)\to(\alpha\to\gamma))$
        \item[f)] $\vdash (\alpha \wedge \beta) \to \alpha$
        \item[g)] $\vdash (\alpha \wedge \beta) \to \beta$
    \end{enumerate}
}
\dem{
    \begin{enumerate}
        \item[a)] $\vdash (\neg \neg \beta \to \neg \alpha) \to (\alpha \to \neg \beta)$ \\
        $
        \begin{array}{lll}
            1. & (\neg \neg \beta \to \neg \alpha) \vdash \alpha \to \neg \beta & TD \\
            2. & (\neg \neg \beta \to \neg \alpha), \alpha \vdash \neg \beta & TD \\
            3. & (\neg \neg \beta \to \neg \alpha) \to (\alpha \to \neg \beta)  & AX_3 \\
            4. & (\neg \neg \beta \to \neg \alpha) & P \\
            5. & \alpha \to \neg \beta & MP(5,4) \\
            6. & \alpha & P \\
            \hline
            7. & \neg \beta & MP(7,6)
        \end{array}
        $
        \vspace*{1em}
        \item[b)] $ \alpha \to \beta, \beta \to \gamma \vdash \alpha \to \gamma $ \\
        $
        \begin{array}{lll}
            1. & (\alpha \to (\beta \to \gamma)) \to ((\alpha \to \beta) \to (\alpha \to \gamma)) & AX_2 \\
            2. & \beta \to \gamma & P \\
            3. & \alpha \to (\beta \to \gamma) & \mbox{Prop 7.1} \\
            4. & (\alpha \to \beta) \to (\alpha \to \gamma) & MP(3,1) \\
            5. & \alpha \to \beta & P \\
            \hline
            6. & \alpha \to \gamma & MP(5,4)
        \end{array}
        $
        \vspace*{1em}
        \item[c)] $(\alpha \to \beta) \vdash \neg \beta \to \neg \alpha$ \\ 
        $
        \begin{array}{lll}
            1. & \neg \neg \alpha \to \alpha & \mbox{Prop 7.4} \\
            2. & \alpha \to \beta & P \\
            3. & \beta \to \neg \neg \beta & \mbox{Prop 7.4} \\
            4. & \neg \neg \alpha \to \beta & \mbox{b) (1,2,3)} \\ 
            5. & \neg \neg \alpha \to \neg \neg \beta & \mbox{\mbox{b) (4,3,5)}} \\
            6. & (\neg \neg \alpha \to \neg \neg \beta) \to (\neg \beta \to \neg \alpha) & AX_3 \\
            \hline \\ 
            7. & \neg \beta \to \neg \alpha & MP(5,6)
        \end{array}
        $
        \vspace*{1em}
        \item[d)] $ \vdash (\beta \to \gamma) \to ((\alpha\to\beta) \to (\alpha\to\gamma)) $ \\
        $
        \begin{array}{lll}
            1. & ( (\beta \to \gamma) \to ( (\alpha \to (\beta \to \gamma)) \to ((\alpha \to \beta) \to (\alpha \to \gamma)) ) ) \to & \\ 
            & ( ( (\beta \to \gamma) \to ( \alpha \to (\beta \to \gamma)) ) \to ( (\beta \to \gamma) \to ((\alpha\to\beta)\to(\alpha\to\gamma )) ) ) & AX_2 \\
            2. & (\alpha \to (\beta \to \gamma)) \to ((\alpha \to \beta) \to (\alpha \to \gamma)) & AX_2 \\
            3. & (\beta \to \gamma) \to ( (\alpha \to (\beta \to \gamma)) \to ((\alpha \to \beta) \to (\alpha \to \gamma)) ) & \mbox{Prop 7.1.} \\ 
            4. & ( (\beta \to \gamma) \to ( \alpha \to (\beta \to \gamma)) ) \to ( (\beta \to \gamma) \to ((\alpha\to\beta)\to(\alpha\to\gamma )) ) & MP(4,1) \\ 
            5. & (\beta \to \gamma) \to ( \alpha \to (\beta \to \gamma)) & AX_1 \\ 
            \hline
            6. & (\beta \to \gamma) \to ((\alpha\to\beta)\to(\alpha\to\gamma )) & MP(5,4)
        \end{array}
        $
        \vspace*{1em}
        \item[e)] $ \vdash (\alpha \to \beta) \to ((\beta\to\gamma)\to(\alpha\to\gamma)) $ \\
        $
        \begin{array}{lll}
            1. & ( (\beta \to \gamma) \to ((\alpha \to \beta) \to (\alpha \to \gamma))) \to & \\
            & \qquad \qquad \qquad (((\beta \to \gamma) \to (\alpha \to \beta)) \to ((\beta \to \gamma) \to (\alpha \to \gamma))) & AX_2 \\
            2. & (\beta \to \gamma) \to ((\alpha \to \beta) \to (\alpha \to \gamma)) & \mbox{(d)} \\
            3. & ((\beta \to \gamma) \to (\alpha \to \beta)) \to ((\beta \to \gamma) \to (\alpha \to \gamma)) & MP(1,2)  \\
            4. & (\alpha \to \beta) \to (((\beta \to \gamma) \to (\alpha \to \beta)) \to ((\beta \to \gamma) \to (\alpha \to \gamma))) & \mbox{Prop 7.1} \\
            5. & (4) \to (((\alpha \to \beta) \to ((\beta \to \gamma) \to (\alpha \to \beta))) \to ((\alpha \to \beta) \to ((\beta \to \gamma) \to (\alpha \to \gamma)))) & AX_2 \\
            6. & ((\alpha \to \beta) \to ((\beta \to \gamma) \to (\alpha \to \beta))) \to ((\alpha \to \beta) \to ((\beta \to \gamma) \to (\alpha \to \gamma))) & MP(5,4) \\
            7. & (\alpha \to \beta) \to ((\beta \to \gamma) \to (\alpha \to \beta)) & AX_1 \\
            \hline
            8. & (\alpha \to \beta) \to ((\beta \to \gamma) \to (\alpha \to \gamma)) & MP(7,6)
        \end{array}
        $
        \vspace*{1em}
        \item[f)] $ \vdash (\alpha \to \beta) \to (\alpha \to \neg \beta) \to \neg \alpha $ \\
        $
        \begin{array}{lll}
            1. & \alpha, \beta \vdash \alpha & TD \\
            2. & \alpha & P
        \end{array}
        $
        \vspace*{1em}
        \item[g)] $ \vdash (\alpha \to \beta) \to (\alpha \to \neg \beta) \to \neg \beta $ \\
        $
        \begin{array}{lll}
            1. & \alpha, \beta \vdash \alpha & TD \\
            2. & \beta & P
        \end{array}
        $
    \end{enumerate}
}

\qs{}{
    Demuestrar sin utilizar $TD$, ni $RA$, los ejercicios d) y e) del item anterior.
}

\dem{
    Se hizo en el problema 2.
}

\qs{}{
    Si se escoge a $\{\neg, \vee \}$ como conjunto completo de conectivos; y como sistema deductivo:\\
    $
    \begin{array}{ll}
        AX_1: & \neg(\alpha \vee \alpha) \vee \alpha \\
        AX_2: & \neg \alpha \vee \alpha \vee \beta\\
        AX_3: & \neg (\alpha \vee \beta) \vee \beta \vee \alpha \\
        AX_4: & \neg(\neg \alpha \vee \beta) \vee \neg (\gamma \vee \alpha) \vee \gamma \vee \beta \\
        MP: & \begin{array}{l}
            \neg \alpha \vee \beta \\
            \alpha \\
            \hline
            \beta 
        \end{array}
    \end{array}
    $
    \\
    Demostrar el teorema $\vdash \neg \alpha \vee \alpha$.
}

\dem{\, \\[1em]
    $
    \begin{array}{lll}
        1. & \neg (\neg (\alpha \vee \alpha) \vee \alpha) \vee \neg (\neg \alpha \vee \alpha \vee \alpha) \vee \neg \alpha \vee \alpha & AX_4 \\
        2. & \neg ( \alpha \vee \alpha ) \vee \alpha & AX_1 \\
        3 & \neg (\neg \alpha \vee \alpha \vee \alpha) \vee \neg \alpha \vee \alpha & MP(2,1) \\
        4.& \neg \alpha \vee \alpha \vee \alpha & AX_2 \\
        \hline
        5. & \neg \alpha \vee \alpha & MP(4,3)
    \end{array}
    $
}
\qs{}{
    Utilizando la siguiente igualdad $p \vee q = \neg p \to q$, muestre que el sistema deductivo anterior se presenta por: \\
    $
    \begin{array}{ll}
        AX_1: & (\alpha \vee \alpha) \to \alpha  \\
        AX_2: & (\alpha \to \alpha) \vee \beta \\
        AX_3: &  (\alpha \vee \beta) \to (\beta \vee \alpha) \\
        AX_4: & (\alpha \vee \beta) \to ((\gamma\vee\alpha)\to(\gamma\vee\alpha)) \\
        MP: & \begin{array}{l}
            \alpha \\
            \alpha \to \beta \\
            \hline
            \beta
        \end{array}
    \end{array}
    $
}

\dem{}

\qs{}{
    Use el sistema deductivo del ejercicio anterior para demostrar el teorma
    \begin{enumerate}
        \item[a)] $\vdash \alpha \to \alpha$
        \item[b)] $\vdash (\alpha \to \beta) \to (\neg \beta \to \neg \alpha) $ 
    \end{enumerate}
}

\dem{
    \begin{enumerate}
        \item[a)] $\vdash \alpha \to \alpha$ \\[1em]
        $
        \begin{array}{lll}
            1. & & 
        \end{array}
        $
    \end{enumerate}
}

\qs{}{
    Si se escoge $\{\neg, \to\}$ como conjunto completo de conectivos; y como sistema deductivo:\\[1em]
    $
    \begin{array}{ll}
        AX_1: & (\alpha \to \beta) \to ((\beta\to\gamma)\to(\alpha\to\gamma)) \\ 
        AX_2: & \alpha \to (\neg \alpha \to \beta) \\ 
        AX_3: & (\neg \alpha \to \alpha) \to \alpha \\
        MP: & \begin{array}{l}
            \alpha \to \beta \\
            \alpha \\
            \hline
            \beta
        \end{array}
    \end{array}
    $\\
    Demostrar el teorema $\vdash \alpha \to \alpha$
}

\dem{\\[1em]
    $
    \begin{array}{lll}
        1. & (\alpha \to (\neg \alpha \to \alpha) ) \to (((\neg \alpha \to \alpha) \to \alpha) \to (\alpha \to \alpha))  & AX_1 \\
        2. & \alpha \to (\neg \alpha \to \alpha)  & AX_2 \\
        3. & ((\neg \alpha \to \alpha) \to \alpha) \to (\alpha \to \alpha) & MP(1,2) \\
        4. & (\neg \alpha \to \alpha) \to \alpha & AX_3 \\
        \hline
        5. & \alpha \to \alpha & MP(4,3)
    \end{array}
    $
}

\end{document}