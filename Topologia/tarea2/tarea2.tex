\documentclass{amsart}
% Language and encoding
\usepackage[spanish]{babel}
\usepackage[utf8]{inputenc}

% margins
% \usepackage[lmargin=2.5cm, rmargin=2cm]{geometry}
\usepackage[lmargin=3cm, rmargin=2cm, bmargin=2cm, tmargin=2cm]{geometry}

% Math packages
\usepackage{amsmath}
\usepackage{amssymb}
\usepackage{amsthm}

% graphics and color
\usepackage{graphicx}
\usepackage{xcolor}
\usepackage{float}
\usepackage{subfigure}
\usepackage{wrapfig}

% color boxes 
\usepackage[most,many,breakable]{tcolorbox}

% fancy headers
\usepackage{fancyhdr}

% Tikz
\usepackage{tikz}
\usetikzlibrary{angles, calc, arrows, arrows.meta}
\AtBeginEnvironment{tikzpicture}{\shorthandoff{>}\shorthandoff{<}}{}{}

% cancel
\usepackage[makeroom]{cancel}
\newcommand\cancelc[2][black]{\renewcommand\CancelColor{\color{#1}}\cancel{#2}}

% fancy settings
\setlength{\headheight}{18pt}
\pagestyle{fancy}
\fancyhf{}
\fancyhead[L]{\includegraphics[height=5mm]{./figures/logo}}
\fancyfoot[R]{\thepage}

% date format
\renewcommand{\datename}{\emph{Fecha:}}


% ------------------ %

% Theorem environments

% colors for theorem environments
\definecolor{myqsbg}{HTML}{f2fbfc}
\definecolor{myqsfr}{HTML}{191971}

% problem environment
\tcbuselibrary{theorems,skins,hooks}
\newtcbtheorem{question}{Problema}
{
	enhanced,
	breakable,
	colback = myqsbg,
	frame hidden,
	boxrule = 0sp,
	borderline west = {2pt}{0pt}{myqsfr},
	sharp corners,
	detach title,
	before upper = \tcbtitle\par\smallskip,
	coltitle = myqsfr,
	fonttitle = \bfseries\sffamily,
	description font = \mdseries,
	separator sign none,
	segmentation style={solid, myqsfr},
}
{th}

% Environment for problems
\newtheorem{problem}{Problema}

% simplified theorem environment

\newcommand{\qs}[2]{\begin{question}{#1}{}#2\end{question}}

% solution command
\newcommand{\sol}[1]{\noindent\textbf{\textit{Solución:}} #1 \par \hfill $_\square$ }
\newcommand{\dem}[1]{\noindent\textbf{\textit{Demostración:}} #1 \par \hfill $_\square$ }

% shortcuts for some math symbols
\newcommand{\R}{\mathbb{R}}
\newcommand{\N}{\mathbb{N}}
\newcommand{\Z}{\mathbb{Z}}
\newcommand{\Q}{\mathbb{Q}}
\newcommand{\C}{\mathbb{C}}

\newcommand{\Bb}{\mathcal{B}}
\newcommand{\ps}[1]{\mathcal{P}\left(#1\right)}

\newcommand{\dd}{\mathrm{d}}
\newcommand{\ZnZ}{\Z / n\Z}

% shortcuts for some math operators
\newcommand{\abs}[1]{\left\lvert #1 \right\rvert}
\newcommand{\norm}[1]{\left\lVert #1 \right\rVert}

\newcommand{\ol}[1]{\overline{#1}}
\usepackage{todonotes}
\title{Tarea 2 - Topología general}
\author{Juan Pablo Lopez, Sebastian Ramirez, Kevin Velez}
\date{Noviembre 2022}

\begin{document}
\maketitle \thispagestyle{fancy}
\qs{Ejercicio 2.b)}{
    Sea $ X =\Z^+ $. Sea $ A \subseteq X $, se define la densidad de $ A $ como el número
    $$ \rho(A) = \lim_{n \to \infty}{ \frac{\abs{ A \cap [n] }}{n} } $$
    donde $ [n] = \{1, 2, \ldots, n\} $ y $ \abs{S} $ denota la cardinalidad de $ S $.

    Sea 
    $$ \tau = \left\{ A \subseteq X | 1 \notin A \vee \rho(A) = 1 \right\} $$
    Caracterice los cerrados en la topología $ \tau $.
}

\dem{
    Los cerrados de $ X $ con la topología $ \tau $ son
    \begin{align*}
        \mathscr{C} &= \left\{ B \subseteq X \, | \, X - B \in \tau \right\} \\ 
        \mathscr{C} &= \left\{ B \subseteq X \, | \, 1 \notin X - B \vee \rho(X - B) = 1 \right\} \\ 
        \mathscr{C} &= \left\{ B \subseteq X \, | \, 1 \notin X \vee 1 \in B \vee \rho(X - B) = 1 \right\} \\
        \intertext{Pero, $ 1 \in X $, entonces}
        \mathscr{C} &= \left\{ B \subseteq X \, | \, 1 \in B \vee \rho(X - B) = 1 \right\} 
    \end{align*}
    Ahora, veamos que conjuntos satisfacen la propiedad de $ \rho(X-B) = 1 $
    \begin{align*}
        1 &= \lim_{n \to \infty}{ \frac{\abs{[n] - B}}{n} } \\ 
        1 &= \lim_{n \to \infty}{ \frac{\abs{[n] - (B \cap [n])}}{n} } \\
        \intertext{Como $B \cap [n] \subseteq [n]$, entonces}
        1 &= \lim_{n \to \infty}{ \frac{\abs{[n]} - \abs{B \cap [n]}}{n} } \\
        1 &= \cancelto{1}{\lim_{n \to \infty}{ \frac{\abs{[n]} }{n}}} - \lim_{n \to \infty}{ \frac{ \abs{B \cap [n] } }{n} } \\
        \cancelc[red]{1} &= \cancelc[red]{1} - \lim_{n \to \infty}{ \frac{ \abs{B \cap [n] } }{n} } \\
        0 &= \lim_{n \to \infty}{ \frac{ \abs{B \cap [n] } }{n} } \\
        0 &= \rho(B)
    \end{align*}
    Por lo tanto, los cerrados de $ X $ con la topología $ \tau $ son
    $$ \mathscr{C} = \left\{ B \subseteq X \, | \, 1 \in B \vee \rho(B) = 0 \right\} $$
}

\newpage
\qs{Ejercicio 4.b)}{
    Sean $m \in \R$ y $B \subseteq \R$. Sean 
    $$ L_B = \left\{ (x,y) \in \R^2 \, | \, y = mx + b \mbox{ para algun $b \in B$} \right\} $$
    y
    $$ \tau \left\{ L_B \, | \, B \subseteq \R \right\} $$

    Demuestre que $ B = L_{ \{ b \}} $ con $ b \in \R$ es un base para $\tau$.
}
\dem{
    Teniendo en cuenta que $ \tau $ es una topología sobre $ \R^2 $, entonces veamos que se cumple la primera condición
    
    $$ \forall \, (x,y) \in \R^2 \, \exists \, L_{\{b\}} \in B \, | \,  (x,y) \in L_{\{b\}} \subseteq \R^2 $$

    sea $ m \in \R $ y $ (x,y) \in \R^2 $, luego esta condición se cumple para $ b = y - mx $

    Asi $ (x,y) \in L_{\{y-mx\}} \in B \subseteq \R^2 $ , ya que se cumple la definición  de $L_{\{b\}}$
    
    $$ L_{b} = \left\{ (x,y) \in \R^2 \, | \, y = mx + b \mbox{ para algun $b \in B$} \right\} $$

    Como $ m,y,x \in \R $ y $ \R $ es cerrado para la suma (resta) y multiplicación, entonces  $ b = y - mx $ pertenece a $\R$ y por ende, para todo $ (x,y) \in \R^2 $  existe un básico que lo contiene.

    Ahora veamos que para todo abierto de la topología $ \tau $ existe una colección de básicos el cual se puede expresar como la union de estos.

    Sea $ L_B \in \tau $ con $ B \subseteq \R $, entonces veamos que

    $$ L_{B} = \bigcup_{b \in B} L_{b} $$

    Sea $ m \in \R $ , tomando $m$ como una pendiente fija, los elementos de $ L_{B} $ son de la forma $ (x,y) = (x, mx + b) $ para algún $ b \in B $, es decir, rectas paralelas con diferentes intersecciones con el eje $y$ como se ilustra en la siguiente figura

    \begin{figure}[H]
        \centering
        \begin{tikzpicture}
            \draw[-{Stealth}] (-0.2,0) -- (3,0) node[right] {$x$};
            \draw[-{Stealth}] (0,-0.2) -- (0,3) node[above] {$y$};
            \clip (-0.4,-0.4) rectangle (3.2,3.2);
            \foreach \x/\n in {0/0, 0.75/1, 1.5/2, 2.25/3}
                \draw[blue] (-1.2,{0.25*(-1) + \x}) -- (3.2,{0.25*(4) + \x}) node[below, black] {\Tiny$l_{\n}\quad$};
            \foreach \x/\n in {0/0, 0.75/1, 1.5/2, 2.25/3}
                \node[rotate=15, blue!70] at (0.75,\x+0.5) {\Tiny$y=mx+b_{\n}$};
        \end{tikzpicture}
    \end{figure}

    Ahora si tomamos $ l_\alpha \in L_{B} $

    $$ l_{\alpha} = \left\{ (x,y) \in \R^2 \, | \, y = mx + \alpha \right\} $$

    luego, existe $ \alpha \in B $ tal que $(x,y) \in L_{\{\alpha\}} \subseteq \bigcup\limits_{b \in B} L_{\{b\}} $

    Asi $ L_{B} \subseteq \bigcup\limits_{b\in B} l_{\{b\}} $

    Ahora observemos la otra inclusion. Sea $ (x,y) \in \bigcup\limits_{b\in B} L_{ \{b\} } $ entonces existe $ j \in B $ tal que $ (x,y) \in L_{\{j\}} $ y se cumple $ y = mx + j $ con $j \in B \subseteq \R $. Pero esta es precisamente la definición de $ L_{B} $ ya que $ L_{B} = \left\{ (x,y) \in \R^2 \, | \, y = mx + b \mbox{ para algún $ b \in B $ } \right\} $

    Asi
    $$ \bigcup_{ b \in B} L_{\{b\}} \subseteq L_{B} $$
    y por lo tanto $ L_{B} = \bigcup\limits_{b \in B} L_{\{b\}} $

    Asi queda demostrado que todo abierto en la topología $ \tau $ se puede generar mediante la unión de una colección de básicos $ L_{\{b\}} \in B $ 

    Por ultimo se demuestra la condición si $ (x,y) \in L_{\{b_{1}\}} $ y $ (x,y) \in L_{\{b_{2}\}} $ entonces existe un básico $ L_{\{b_{3}\}} $ tal que $ (x,y) \in L_{\{b_{3}\}} \subseteq L_{\{b_{1}\}} \cap L_{\{b_{2}\}} $

    Pero esto solo sucede si
    $$  L_{\{b_1\}} = L_{\{b_2\}} \Longleftrightarrow b_1 = b_2$$
    ya que son rectas paralelas, de otro modo, su intersección sería vacío, el cual también es un abierto de la topología $ \tau $.
}

\qs{Ejercicio 11.b)}{
    Siendo $X=[0,1]\times [0,1]$, con la topologia del orden lexicografico hallar la clausura de $B=\{(1-\frac{1}{n},\frac{1}{2}:\ n\in \Z^{+} \}$.
}
\dem{
    Como X tiene elemento maximo $(1,1)$ y minimo $(0,0)$, una base para la topologia son los intervalos cerrados a ambos lados. Por el teorema 17.6 en Munkres se tiene $\overline{B}=B\cup B'$, donde B' son los puntos de acumulación de B. Vamos a probar que $B'=\emptyset$, para ello tomemos un elemento $(a,b)$ con $0<a\leq 1$ y veamos que no puede ser un punto de acumulación de B. Supongamos que $a\neq \frac{1}{n}$ para todo $n\in \Z^+$, por propiedad Arquimediana existe un natural n tal que $\frac{1}{n+1}<a<\frac{1}{n}$ (es posible ya que $a>0$) y con ello vemos que el intervalo $[(a-\delta,b),(a+\delta,b)]$ con $\delta=\frac{1}{2} \min \{|a-\frac{1}{n+1}|,|a-\frac{1}{n}| \}$ es abierto y contiene a $(a,b)$ pero no intercepta a $B-\{(a,b)\}=B$ debido a su misma construcción, es decir, $(a,b)$ no es un punto de acumulación. Por otro lado, si $a=\frac{1}{n}$ para algún natural n, ahora tomamos   el intervalo $[(a,0),(a,1)]$ si $b=\frac{1}{2}$  ya que solo contiene elementos de la forma $(\frac{1}{n},x)$ con $x\in [0,1]$ con un n fijo y por lo tanto no intercepta a $B-\{(a,b)\}$. Si $b\neq \frac{1}{2}$ podemos asumir sin perdida de generalidad que $b<\frac{1}{2}$ y hacer $\delta=\frac{1}{2}(\frac{1}{2}-b)$, tendría que el intervalo $[(a,b),(a,b+\delta)]$ contiene a $(a,b)$ pero no contiene a ningún elemento de B ya que los elementos de B tienen segunda coordenada igual a $\frac{1}{2}$.

    Ahora veamos que pasa con $a=0$, vemos que el intervalo $[(0,0),(0,1)]$ contiene a $(a,b)$ pero no tienen a ningún elemento de B ya que son mayores a $(0,x)$ con $x\in [0,1]$.\\
    Finalmente, se concluye que B no tiene puntos limite y por lo tanto $\overline{B}=B\cup B'=B$ , es decir , B es cerrado.
}
\end{document}