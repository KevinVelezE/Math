\documentclass{amsart}
% Language and encoding
\usepackage[spanish]{babel}
\usepackage[utf8]{inputenc}

% margins
% \usepackage[lmargin=2.5cm, rmargin=2cm]{geometry}
\usepackage[lmargin=3cm, rmargin=2cm, bmargin=2cm, tmargin=2cm]{geometry}

% Math packages
\usepackage{amsmath}
\usepackage{amssymb}
\usepackage{amsthm}
\usepackage{mathrsfs}

% graphics and color
\usepackage{graphicx}
\usepackage{xcolor}
\usepackage{float}
\usepackage{subfigure}
\usepackage{wrapfig}

% color boxes 
\usepackage[most,many,breakable]{tcolorbox}

% fancy headers
\usepackage{fancyhdr}

% Tikz
\usepackage{tikz}
\usetikzlibrary{angles, calc, arrows, arrows.meta}
\AtBeginEnvironment{tikzpicture}{\shorthandoff{>}\shorthandoff{<}}{}{}

% cancel
\usepackage[makeroom]{cancel}
\newcommand\cancelc[2][black]{\renewcommand\CancelColor{\color{#1}}\cancel{#2}}

% fancy settings
\setlength{\headheight}{18pt}
\pagestyle{fancy}
\fancyhf{}
\fancyhead[L]{\includegraphics[height=5mm]{./figures/logo}}
\fancyfoot[R]{\thepage}

% date format
\renewcommand{\datename}{\emph{Fecha:}}


% ------------------ %

% Theorem environments

% colors for theorem environments
\definecolor{myqsbg}{HTML}{f2fbfc}
\definecolor{myqsfr}{HTML}{191971}

% problem environment
\tcbuselibrary{theorems,skins,hooks}
\newtcbtheorem{question}{Problema}
{
	enhanced,
	breakable,
	colback = myqsbg,
	frame hidden,
	boxrule = 0sp,
	borderline west = {2pt}{0pt}{myqsfr},
	sharp corners,
	detach title,
	before upper = \tcbtitle\par\smallskip,
	coltitle = myqsfr,
	fonttitle = \bfseries\sffamily,
	description font = \mdseries,
	separator sign none,
	segmentation style={solid, myqsfr},
}
{th}

% simplified theorem environment
\newcommand{\qs}[2]{\begin{question}{#1}{}#2\end{question}}

% Environment for problems
\newtheorem{problem}{Problema}

% solution command
\newcommand{\sol}[1]{\noindent\textbf{\textit{Solución:}} #1 \par \hfill $_\square$ }
\newcommand{\dem}[1]{\noindent\textbf{\textit{Demostración:}} #1 \par \hfill $_\square$ }

% shortcuts for some math symbols
\newcommand{\R}{\mathbb{R}}
\newcommand{\N}{\mathbb{N}}
\newcommand{\Z}{\mathbb{Z}}
\newcommand{\Q}{\mathbb{Q}}
\newcommand{\C}{\mathbb{C}}

\newcommand{\Bb}{\mathcal{B}}
\newcommand{\ps}[1]{\mathcal{P}\left(#1\right)}

\newcommand{\dd}{\mathrm{d}}
\newcommand{\ZnZ}{\Z / n\Z}

% shortcuts for some math operators
\newcommand{\abs}[1]{\left\lvert #1 \right\rvert}
\newcommand{\norm}[1]{\left\lVert #1 \right\rVert}

\newcommand{\ol}[1]{\overline{#1}}

\title{Topología}
\author{Kevin Velez}
\date{\today}
\usepackage{lipsum}
\begin{document}

\maketitle \thispagestyle{fancy}

\qs{Problema en clase}{
    Demostrar que $d$ es una métrica en $\Z^+$, donde $d$ está dada por:
    
    \begin{equation*}
        d(n,m) = \abs{\frac{1}{n} - \frac{1}{m}}
    \end{equation*}
}
\dem{
    Debemos mostrar que $d$ satisface las cuatro propiedades para ser una métrica. Sean $n,m,k \in \Z^+$.
    
    \begin{enumerate}
        \item[i)] $$ d(n,n) = \abs{\frac{1}{n} - \frac{1}{n}} = \abs{0} = 0 $$
        \item[ii)] $ d(n,m) \geq 0$ por definición, ya que el valor absoluto es no negativo. 
        \item[iii)] $$ d(n,m) = \abs{\frac{1}{n} - \frac{1}{m}} = \abs{(-1)\left( \frac{1}{m} - \frac{1}{n} \right)} = \abs{\frac{1}{m} - \frac{1}{n}} = d(m,n)$$
        \item[iv)] $$ d(n,m) = \abs{\frac{1}{n} - \frac{1}{m}} = \abs{\frac{1}{n} - \frac{1}{k} + \frac{1}{k} - \frac{1}{m}} \leq \abs{\frac{1}{n} - \frac{1}{k}} + \abs{\frac{1}{k} - \frac{1}{m}} = d(n,k) + d(k,m) $$ 
    \end{enumerate}

    Por tanto, $d$ es una métrica.
}

\qs{Ejercicio 5(c). \emph{Elementos de topología - Garcia y Dal Lago} }{
    Demostrar que $ A = \left\{ (x,y) \in \R^2: \; 0 < x < 1, \;  0 < y < 1, \;  x \neq \frac{1}{n} \;  \forall n \in \N \right\} \subseteq \R^2$ es un abierto de $\R^2$ con la métrica euclídea. 
}
\dem{
    \begin{figure}[H]
    \begin{center}
        \begin{tikzpicture}[scale=4.5]
            \draw[-{Latex}, dashed] (-0.1,0) -- (1.2,0);
            \draw[-{Latex}, dashed] (0,-0.1) -- (0,1.2);
            \draw[dashed] (1,0) -- (1,1);
            \draw[dashed] (0,1) -- (1,1);
            \node[below right] at (1.2,0) {\footnotesize $x$};
            \node[above left] at (0,1.2) {\footnotesize $y$};
            
            \fill[gray!20, opacity=0.5] (0,0) rectangle (1,1);

            \draw[dashed] (0.5,0) -- (0.5,1);
            \draw[dashed] ({1/3},0) -- ({1/3},1);
            \draw[dashed] ({1/4},0) -- ({1/4},1);
            \draw[dashed] ({1/5},0) -- ({1/5},1);
            \node at (0.12,0.5) {\tiny $\cdots$};

            \node[below] at (0.5,0) {\tiny $\frac{1}{2}$};
            \node[below] at ({1/3},0) {\tiny $\tfrac{1}{3}$};
            \node[below] at ({1/4},0) {\tiny $\tfrac{1}{4}$};
            
            \coordinate (x) at (0.6,0.3);
            \fill (x) circle (0.25pt) node[above right] {\tiny $(x,y)$};

            
            \draw[-{Latex}|, blue] (x) -- (0.6, 0);
            \draw[-{Latex}|, blue ] (x) -- (0.5,0.3);
            \draw[-{Latex}|, blue ] (x) -- (0.6,1);
            \draw[-{Latex}|, blue ] (x) -- (1,0.3);

            \node[below] at (0.8,0.3) {\tiny $d_3$};
            \node[below] at (0.55,0.3) {\tiny $d_2$};
            \node[right] at (0.6,0.65) {\tiny $d_1$};
            \node[right] at (0.6,0.15) {\tiny $d_4$};
            
            % \node at (1.2, 0.65) {\tiny $d_1 = 1-y$};
            % \node at (1.2, 0.55) {\tiny $d_2 = x-\frac{1}{k+1}$};
            % \node at (1.2, 0.45) {\tiny $d_3 = y$};
            % \node at (1.2, 0.35) {\tiny $d_4 = \frac{1}{n} - x$};

            \node at (1.2, 0.5) {\tiny
                $\begin{array}{ll}
                    d_1 = &  \hspace*{-2.5mm} 1-y \\[1mm]
                    d_2 = & \hspace*{-2.5mm} x - \frac{1}{k+1} \\[1mm]
                    d_3 = & \hspace*{-2.5mm} \frac{1}{k}-x \\[1mm]
                    d_4 = & \hspace*{-2.5mm} y
                \end{array}$
            };
        \end{tikzpicture}
    \end{center}
    \end{figure}

    Sea $(x,y) \in A$, vamos a mostrar que existe $r>0$ tal que $B((x,y), r) \subseteq A$. Tenemos entonces que $ 0<x<1$, por lo que $ \frac{1}{x} > 1$, existe un $k \in \N$ tal que $k < \frac{1}{x} < k + 1$, luego $\frac{1}{k+1} < x < \frac{1}{k}$, además, $ 0 < y < 1$. Consideramos entonces el radio $r = \min\left\{ 1-y, x - \frac{1}{k+1}, \frac{1}{k} - x, y \right\} > 0$, y la bola abierta $B((x,y), r)$.

    Sea $(u,v) \in B((x,y), r)$, entonces, como la métrica es la euclídea, se tiene que
    
    \begin{align}
        d((x,y), (u,v)) &< r \nonumber \\ 
        \sqrt{ (x-u)^2 + (y-v)^2 } &< r \nonumber \\
        (x-u)^2 + (y-v)^2 &< r^2 \label{eq:euc}
    \end{align}
    
    Como $(y-u)^2 > 0$, entonces $(x-u)^2 < r^2$, así

    \begin{align*}
        (x-u)^2 &< r^2 \\
        \abs{x-u} &< \abs{r} \\
        \abs{x-u} &< r \\
        -r < x-u &< r \\
        -r < u-x &< r \\
    \end{align*}

    por definición, $ r \leq \frac{1}{k} - x$, y $r \leq x - \frac{1}{k+1}$, por lo que $-r \geq \frac{1}{k+1} - x$, entonces

    \begin{align*}
        \frac{1}{k+1} - x \leq -r < u - x &< r \leq \frac{1}{k} - x \\
        \frac{1}{k+1} - \cancelc[red]{x} < u - \cancelc[red]{x} &< \frac{1}{k} - \cancelc[red]{x} \\
        \frac{1}{k+1} < u &< \frac{1}{k} \\
    \end{align*}

    Análogamente, en \eqref{eq:euc}, como $(x-u)^2 > 0$, entonces $(y-v)^2 < r^2$, así

    \begin{align*}
        (y-v)^2 &< r^2 \\
        \abs{y-v} &< \abs{r} \\
        \abs{y-v} &< r \\
        -r < y-v &< r \\
        -r < v-y &< r \\
    \end{align*}

    por definición, $ r \leq 1-y$, y $r \leq y$, por lo que $-r \geq -y$, entonces

    \begin{align*}
        -y \leq -r < v-y &< r \leq 1-y \\
        -\cancelc[red]{y} < v - \cancelc[red]{y} &< 1-\cancelc[red]{y} \\
        0 < v  &< 1 \\
    \end{align*}

    Así, tenemos que $0 < v < 1$ y $\frac{1}{k+1} < u < \frac{1}{k}$, por lo que $(u,v) \in A$, y por tanto $B((x,y), r) \subseteq A$. Luego, $A$ es un abierto de $\R^2$
}
\newpage
\qs{Ejercicio 8. \emph{Elementos de topología - Garcia y Dal Lago}}{
    Sea $E$ un espacio métrico, $A \subseteq E$ y $x \in E$. Se define la distancia de $x$ a $A$ por

    $$ d(x,A) = \inf \left\{ d(x,y) : y \in A \right\} $$

    \begin{enumerate}
        \item[a)] Demostrar que si definimos $\delta: E \to \R$ por $\delta(x)=d(x,A)$, $\delta$ es continua.
        \item[b)] Probar que si $r>0$, $\left\{ x : d(x,A)\leq r \right\}$ es cerrado
    \end{enumerate}
}

\dem{
    \begin{enumerate}
        \item[a)] Sea $x, x_0 \in E$, y $a \in A$, entonces $d(x, a) \leq d(x, x_0) + d(x_0, a) $, como $d(x,A) \leq d(x,a)$ por ser el ínfimo, entonces $d(x,A) \leq d(x, x_0) + d(x_0, a)$, así, se tiene que 
        
        $$ \delta(x) \leq d(x, x_0) + d(x_0, a) $$
        
        Esto se cumple para todo para todo $a \in A$, en particular, para el ínfimo. Como 
        
        $\inf \left\{ d(x, x_0) + d(x_0, a) : a \in A \right\} = d(x,x_0) + \inf \left\{ d(x_0, a) : a \in A \right\} = d(x,x_0) + \delta(x_0)$. Por lo que
        
        \begin{align*}
            \delta(x) \leq d(x, x_0) + \delta(x_0) \\
            \delta(x) - \delta(x_0) \leq d(x, x_0)
        \end{align*}

        De manera análoga, partiendo de $d(x_0, a) \leq d(x_0, x) + d(x, a)$, se llega a que $\delta(x) - \delta(x_0) \geq - d(x_0, x)$, por lo que
        $$ \abs{\delta(x) - \delta(x_0)} \leq d(x, x_0) $$

        Ahora, sea $\varepsilon > 0$, podemos tomar $x_0$ $\varepsilon$-cercano a $x$, de tal manera que $ d(x, x_0) = \varepsilon $, en otras palabras, tomamos $r = \varepsilon$ y tenemos que $d(x, x_o) \leq r = \varepsilon$ implica que $ \abs{\delta(x) - \delta(x_0)} \leq d(x, x_0) \leq r = \varepsilon $. Lo cual quiere decir, que la función $\delta$ es continua para todo punto $x \in E$

        \item[b)] Sea $r > 0$, vamos a probar que $ B = \left\{ x : d(x,A) \leq r \right\}$ es cerrado. Notemos que $ B = \delta^{-1}\left( [0,1] \right) $. Como $\delta$ es continua, entonces preimagen de cerrados es cerrado, y como $[0,1]$ es cerrado, entonces $B$ es cerrado. 
    \end{enumerate}
}

\qs{Ejercico 6. \emph{Topología - Munkres}}{
    Pruebe que las topologías de $\R_\ell$ y $\R_K$ no son comparables.
}
\dem{
    Sean $\tau_\ell$ y $\tau_K$ las topologías de $\R_\ell$ y $\R_K$ respectivamente. Consideremos el elemento básico $[0,1)$ para $\tau_\ell$. No hay ningún básico para $\tau_K$ que contenga al 0, y este contenido en $[0,1)$, así que $\tau_\ell \cancel{\subset} \tau_K$.
    
    Ahora, consideremos consideremos el básico $(-1,1) \setminus K$ para $\tau_K$, No existe ningún básico para $\tau_\ell$ que contenga al 0, y este contenido en $(-1,1) \setminus K$, así que $\tau_K \cancel{\subset} \tau_\ell$.

    Por lo tanto, $\tau_\ell$ y $\tau_K$ no son comparables.
}

\end{document}