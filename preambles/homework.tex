% Language and encoding
\usepackage[spanish]{babel}
\usepackage[utf8]{inputenc}

% margins
% \usepackage[lmargin=2.5cm, rmargin=2cm]{geometry}
\usepackage[lmargin=3cm, rmargin=2cm, bmargin=2cm, tmargin=2cm]{geometry}

% Math packages
\usepackage{amsmath}
\usepackage{amssymb}
\usepackage{amsthm}
\usepackage{mathrsfs}

% graphics and color
\usepackage{graphicx}
\usepackage{xcolor}
\usepackage{float}
\usepackage{subfigure}
\usepackage{wrapfig}

% color boxes 
\usepackage[most,many,breakable]{tcolorbox}

% fancy headers
\usepackage{fancyhdr}

% Tikz
\usepackage{tikz}
\usetikzlibrary{angles, calc, arrows, arrows.meta}
\AtBeginEnvironment{tikzpicture}{\shorthandoff{>}\shorthandoff{<}}{}{}

% cancel
\usepackage[makeroom]{cancel}
\newcommand\cancelc[2][black]{\renewcommand\CancelColor{\color{#1}}\cancel{#2}}

% fancy settings
\setlength{\headheight}{18pt}
\pagestyle{fancy}
\fancyhf{}
\fancyhead[L]{\includegraphics[height=5mm]{./figures/logo}}
\fancyfoot[R]{\thepage}

% date format
\renewcommand{\datename}{\emph{Fecha:}}


% ------------------ %

% Theorem environments

% colors for theorem environments
\definecolor{myqsbg}{HTML}{f2fbfc}
\definecolor{myqsfr}{HTML}{191971}

% problem environment
\tcbuselibrary{theorems,skins,hooks}
\newtcbtheorem{question}{Problema}
{
	enhanced,
	breakable,
	colback = myqsbg,
	frame hidden,
	boxrule = 0sp,
	borderline west = {2pt}{0pt}{myqsfr},
	sharp corners,
	detach title,
	before upper = \tcbtitle\par\smallskip,
	coltitle = myqsfr,
	fonttitle = \bfseries\sffamily,
	description font = \mdseries,
	separator sign none,
	segmentation style={solid, myqsfr},
}
{th}

% simplified theorem environment
\newcommand{\qs}[2]{\begin{question}{#1}{}#2\end{question}}

% Environment for problems
\newtheorem{problem}{Problema}

% solution command
\newcommand{\sol}[1]{\noindent\textbf{\textit{Solución:}} #1 \par \hfill $_\square$ }
\newcommand{\dem}[1]{\noindent\textbf{\textit{Demostración:}} #1 \par \hfill $_\square$ }

% shortcuts for some math symbols
\newcommand{\R}{\mathbb{R}}
\newcommand{\N}{\mathbb{N}}
\newcommand{\Z}{\mathbb{Z}}
\newcommand{\Q}{\mathbb{Q}}
\newcommand{\C}{\mathbb{C}}

\newcommand{\Bb}{\mathcal{B}}
\newcommand{\ps}[1]{\mathcal{P}\left(#1\right)}

\newcommand{\dd}{\mathrm{d}}
\newcommand{\ZnZ}{\Z / n\Z}

% shortcuts for some math operators
\newcommand{\abs}[1]{\left\lvert #1 \right\rvert}
\newcommand{\norm}[1]{\left\lVert #1 \right\rVert}

\newcommand{\ol}[1]{\overline{#1}}