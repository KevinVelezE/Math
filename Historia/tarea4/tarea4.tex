\documentclass{amsart}
% Language and encoding
\usepackage[spanish]{babel}
\usepackage[utf8]{inputenc}

% margins
% \usepackage[lmargin=2.5cm, rmargin=2cm]{geometry}
\usepackage[lmargin=3cm, rmargin=2cm, bmargin=2cm, tmargin=2cm]{geometry}

% Math packages
\usepackage{amsmath}
\usepackage{amssymb}
\usepackage{amsthm}

% graphics and color
\usepackage{graphicx}
\usepackage{xcolor}
\usepackage{float}
\usepackage{subfigure}
\usepackage{wrapfig}

% color boxes 
\usepackage[most,many,breakable]{tcolorbox}

% fancy headers
\usepackage{fancyhdr}

% Tikz
\usepackage{tikz}
\usetikzlibrary{angles, calc, arrows, arrows.meta}
\AtBeginEnvironment{tikzpicture}{\shorthandoff{>}\shorthandoff{<}}{}{}

% cancel
\usepackage[makeroom]{cancel}
\newcommand\cancelc[2][black]{\renewcommand\CancelColor{\color{#1}}\cancel{#2}}

% fancy settings
\setlength{\headheight}{18pt}
\pagestyle{fancy}
\fancyhf{}
\fancyhead[L]{\includegraphics[height=5mm]{./figures/logo}}
\fancyfoot[R]{\thepage}

% date format
\renewcommand{\datename}{\emph{Fecha:}}


% ------------------ %

% Theorem environments

% colors for theorem environments
\definecolor{myqsbg}{HTML}{f2fbfc}
\definecolor{myqsfr}{HTML}{191971}

% problem environment
\tcbuselibrary{theorems,skins,hooks}
\newtcbtheorem{question}{Problema}
{
	enhanced,
	breakable,
	colback = myqsbg,
	frame hidden,
	boxrule = 0sp,
	borderline west = {2pt}{0pt}{myqsfr},
	sharp corners,
	detach title,
	before upper = \tcbtitle\par\smallskip,
	coltitle = myqsfr,
	fonttitle = \bfseries\sffamily,
	description font = \mdseries,
	separator sign none,
	segmentation style={solid, myqsfr},
}
{th}

% Environment for problems
\newtheorem{problem}{Problema}

% simplified theorem environment

\newcommand{\qs}[2]{\begin{question}{#1}{}#2\end{question}}

% solution command
\newcommand{\sol}[1]{\noindent\textbf{\textit{Solución:}} #1 \par \hfill $_\square$ }
\newcommand{\dem}[1]{\noindent\textbf{\textit{Demostración:}} #1 \par \hfill $_\square$ }

% shortcuts for some math symbols
\newcommand{\R}{\mathbb{R}}
\newcommand{\N}{\mathbb{N}}
\newcommand{\Z}{\mathbb{Z}}
\newcommand{\Q}{\mathbb{Q}}
\newcommand{\C}{\mathbb{C}}

\newcommand{\Bb}{\mathcal{B}}
\newcommand{\ps}[1]{\mathcal{P}\left(#1\right)}

\newcommand{\dd}{\mathrm{d}}
\newcommand{\ZnZ}{\Z / n\Z}

% shortcuts for some math operators
\newcommand{\abs}[1]{\left\lvert #1 \right\rvert}
\newcommand{\norm}[1]{\left\lVert #1 \right\rVert}

\newcommand{\ol}[1]{\overline{#1}}

\title{TAREA 3}
\author{Kevin Rincon, Diego Tarapuez, Kevin Velez}
\date{Octubre 2022}
\fancyhead[R]{\footnotesize SEMINARIO DE HISTORIA DE LA MATEMÁTICA}

% rosa1 #e13c6c
\definecolor{rs1}{rgb}{0.88,0.23,0.42}
% rosa2 #eb657f
% \definecolor{rs2}{rgb}{0.92,0.39,0.50}
% azul1 #062149
%\definecolor{az1}{rgb}{0.36,0.10,0.30}
\definecolor{az1}{rgb}{0.9,0,0}
% \definecolor{az1}{rgb}{0.02,0.13,0.29}
% azul2 #5f77a6
\definecolor{az2}{rgb}{0.37,0.47,0.65}
% azul3 #91a0c6
\definecolor{az3}{rgb}{0.57,0.63,0.78}
% % morado1 #5c1a4d
% \definecolor{mr1}{rgb}{0.36,0.10,0.30}
% verde #16825D
\definecolor{mr1}{rgb}{0.09,0.51,0.36}
% naranja #FF9800
\definecolor{rs2}{rgb}{1.00,0.60,0.00}


\begin{document}
\maketitle \thispagestyle{fancy}

\newcommand{\pabt}{\textcolor{rs1}{ABT}}

\newcommand{\kt}{\textcolor{az2}{KT}}
\newcommand{\kb}{\textcolor{az2}{KB}}
\newcommand{\bt}{\textcolor{az2}{BT}}

\newcommand{\az}{\textcolor{az1}{AZ}}
\newcommand{\ak}{\textcolor{az1}{AK}}
\newcommand{\kz}{\textcolor{az1}{KZ}}

\newcommand{\tz}{\textcolor{az1}{TZ}}
\newcommand{\te}{\textcolor{az1}{TE}}
\newcommand{\ez}{\textcolor{az1}{EZ}}

\newcommand{\de}{\textcolor{mr1}{DE}}
\newcommand{\db}{\textcolor{mr1}{DB}}
\newcommand{\be}{\textcolor{mr1}{BE}}

\newcommand{\fm}{\textcolor{rs2}{FM}}
\newcommand{\fo}{\textcolor{rs2}{FO}}
\newcommand{\om}{\textcolor{rs2}{OM}}

\qs{La cuadratura de la parábola por medios mecánicos.}{
    Sea \textcolor{rs1}{$ABT$} el segmento de la parábola limitado por la recta $AT$ y la parábola \textcolor{rs1}{$ABT$}, $D$ el punto medio de $AT$, $B$ el vértice de la parábola, la recta \textcolor{mr1}{$DB$} es el eje de la parábola. $\tz$ es tangente a la parábola en $T$ y $\az$ es perpendicular a $AT$, Sea $F$ un punto en $AT$, se traza $\fm$ perpendicular a $AT$. Demostrar que $B$ es el punto medio de $\kt$ y de $\de$, y demostrar que $\frac{AT}{AF} = \frac{\fm}{\fo}$
}

\dem{
    De manera análitica, podemos la parábola en el plano, tomando el eje $x$ como la recta $AT$ y el eje $y$ como la recta $\db$, la parábola está descrita entonces por la función $\textcolor{rs1}{y=-ax^2+b}$, con $a,b > 0$, por lo que $\db = b$. De está manera, el corte con el eje $x$ se dan en $\left\{ -\sqrt{\frac{b}{a}}, \sqrt{\frac{b}{a}} \right\}$.
    \begin{figure}[H]
        \centering
        \renewcommand{\a}{0.25}
        \renewcommand{\b}{1}
        \newcommand{\sqab}{(\b/\a)^(1/2)}

        \begin{tikzpicture}[scale=1.5]
            \coordinate (A) at ({-\sqab},0);
            \coordinate (T) at ({\sqab},0);
            \coordinate (Z) at ({-\sqab},4*\b);
            \coordinate (E) at (0, 2*\b);
            \coordinate (B) at (0,\b);
            \coordinate (K) at ({-\sqab},2*\b);
            \coordinate (D) at (0,0);
            \coordinate (F) at ({-\sqab/2},0);
            \coordinate (O) at ({-\sqab/2},{-\a*(\sqab)*(\sqab)/4 + \b});
            \coordinate (M) at ({-\sqab/2},{-((2*\b)/(\sqab))*(-\sqab/2 - \sqab)});

            \draw[{Stealth}-{Stealth}, gray, dashed] ({-\sqab - 1},0) -- ({\sqab + 1},0);
            \draw[-{Stealth}, gray, dashed] (0,0) -- (0,4*\b + 0.5);
            \node[gray, above] at (0,4*\b + 0.5) {$y$};
            \node[gray, right] at ({\sqab + 1},0) {$x$};

            \draw[very thick, domain=-2:2, rs1] plot (\x, -\a*\x*\x + \b);
            \draw[very thick] (A) -- (T);
            \draw[very thick, az1] (A) -- (Z) -- (T);
            \draw[very thick, az2] (T) -- (K);
            \draw[very thick, mr1] (D) -- (E);
            \draw[very thick, rs2] (F) -- (M);
            \draw[very thick, dashed] (A) -- (B);
            
            
            % points
            \begin{scope}[fill=black,opacity=0.5]
                \fill (A) circle (1pt) node[below left] {$A$};
                \fill (T) circle (1pt) node[below right] {$T$};
                \fill (Z) circle (1pt) node[above left] {$Z$};
                \fill (E) circle (1pt) node[above right] {$E$};
                \fill (B) circle (1pt) node[above right] {$B$};
                \fill (K) circle (1pt) node[left] {$K$};
                \fill (D) circle (1pt) node[below] {$D$};
                \fill (F) circle (1pt) node[below] {$F$};
                \fill (O) circle (1pt) node[above left] {$O$};
                \fill (M) circle (1pt) node[below left] {$M$};
            \end{scope}
        \end{tikzpicture}
    \end{figure}

    Como $\tz$ es tangente a la parábola en el punto $T = \sqrt{\frac{b}{a}}$, entonces, derivando y evaluando en el punto, obtenemos la pendiente de la recta $\tz$.
    \begin{align*}
        \frac{d}{dx}(-ax^2 + b)\Big|_{_{x=\sqrt{\frac{b}{a}}}} = -2 a x \Big|_{_{x=\sqrt{\frac{b}{a}}}} = -2 a \sqrt{\frac{b}{a}} = -2 \sqrt{ab}
    \end{align*}
    por lo que la ecuación de la recta $\tz$ es
    $$ \textcolor{az1}{y = -2\sqrt{ab} \cdot x + 2b} $$

    Por lo qué, el punto $E$, se encuentra al remplazar $0$ en la ecuación de la recta $\tz$. dando como resultado $ E = (0, 2b) $, aquí queda evidente entonces que $B$ es el punto medio de $\de$, pues $\db=b$ y $\de=2b$.

    Ahora, hallamos la recta $\kt$ que pasa por $T$ y $B$, primero la pendiente
    $$ m = -\frac{b}{\sqrt{\frac{b}{a}}} = -b\sqrt{\frac{a}{b}} = -\sqrt{ab} $$
    Entonces, la ecuación de la recta $\kt$ es
    $$ \textcolor{az2}{y = -\sqrt{ab}\cdot x + b } $$

    Por lo tanto, la coordenada $y$ del punto $K$ se encuentra al remplazar $x$ en la ecuación de la recta $\kt$, dando como resultado $ K = (-\sqrt{\frac{b}{a}}, 2b) $.

    Verifiquemos entonces que la distancia de $K$ a $B$ es igual a la distancia de $B$ a $T$.

    \begin{align*}
        d(K,B) &= \sqrt{ \left(-\sqrt{\frac{b}{a}} - 0\right)^2 + \left( 2b - b \right)^2 } & 
        d(B,T) &= \sqrt{ \left(0 - \sqrt{\frac{b}{a}}\right)^2 + \left( b - 0 \right)^2 }  \\
        d(K,B) &= \sqrt{ \left(\sqrt{\frac{b}{a}} \right)^2 +  b^2 } &
        d(B,T) &= \sqrt{ \left(\sqrt{\frac{b}{a}} \right)^2 + b^2 } \\ 
    \end{align*}
    Por lo tanto, $d(K,B) = d(B,T)$, es decir, $B$ es el punto medio de $\kt$.


    Ahora, veamos que si $F$ tiene coordenadas $(x,0)$, entonces $O$ tiene coordenadas $(x, -ax^2+b)$ y $M$ tiene coordenadas $(x, -2\sqrt{ab}x + 2b)$, entonces veamos la razón $\frac{\fm}{\fo}$
    $$ \frac{\fm}{\fo} = \frac{-2\sqrt{ab}+2b}{-ax^2 + b } $$
    Y veamos la razón $\frac{AT}{AF}$
    $$ \frac{AT}{AF} = \frac{ 2 \sqrt{\frac{b}{a}} }{x + 2 \sqrt{\frac{b}{a}}} $$
    Y sabemos que $\frac{\fm}{\fo} = \frac{AT}{AF}$ si y solo si $\fm \cdot AF = \fo \cdot AT$, verifiquemos entonces

    \begin{align*}
        \fm \cdot AF &= \left(-2\sqrt{ab}x+2b\right)\left( x + \sqrt{\frac{b}{a}} \right) &
        \fo \cdot AT &= \left(-ax^2 + b\right)\left( 2 \sqrt{\frac{b}{a}} \right) \\
        \fm \cdot AF &= -2\sqrt{ab}x^2 - \cancelc[red]{2bx} + \cancelc[red]{2bx} + 2b \sqrt{\frac{b}{a}} &
        \fo \cdot AT &= -2\sqrt{ab}x^2 + 2b\sqrt{\frac{b}{a}} \\
        \fm \cdot AF &= -2\sqrt{ab}x^2 + 2b \sqrt{\frac{b}{a}} & \\
    \end{align*}

    Como podemos ver, efectivamente $\fm \cdot AF = \fo \cdot AT$, por lo que se tiene que $\frac{\fm}{\fo} = \frac{AT}{AF}$

    Por último, veamos que los triángulos $\triangle ABT$ y $\triangle AZT$, comparten la misma base, la altura del triángulo $\triangle ABT$ es el segmento $\db$, por lo que su altura es $b$, y la altura del triángulo $\triangle AZT$ es el segmento $\az$, por lo que su altura $4b$, de este modo, el área del triángulo $\triangle ABT$ es $\frac{1}{2} \cdot AT \cdot b$ y la medida del triángulo $\triangle AZT$ es $\frac{1}{2} \cdot  AT \cdot 4b $, y por tanto, el triángulo $\triangle AZT$ es cuatro veces mayor en área (medida) que el triángulo $\triangle ABT$, es decir
    $$ \triangle AZT = 4 \triangle ABT $$
}

\end{document}