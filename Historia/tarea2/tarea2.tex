\documentclass{amsart}
% Language and encoding
\usepackage[spanish]{babel}
\usepackage[utf8]{inputenc}

% margins
% \usepackage[lmargin=2.5cm, rmargin=2cm]{geometry}
\usepackage[lmargin=3cm, rmargin=2cm, bmargin=2cm, tmargin=2cm]{geometry}

% Math packages
\usepackage{amsmath}
\usepackage{amssymb}
\usepackage{amsthm}

% graphics and color
\usepackage{graphicx}
\usepackage{xcolor}
\usepackage{float}
\usepackage{subfigure}
\usepackage{wrapfig}

% color boxes 
\usepackage[most,many,breakable]{tcolorbox}

% fancy headers
\usepackage{fancyhdr}

% Tikz
\usepackage{tikz}
\usetikzlibrary{angles, calc, arrows, arrows.meta}
\AtBeginEnvironment{tikzpicture}{\shorthandoff{>}\shorthandoff{<}}{}{}

% cancel
\usepackage[makeroom]{cancel}
\newcommand\cancelc[2][black]{\renewcommand\CancelColor{\color{#1}}\cancel{#2}}

% fancy settings
\setlength{\headheight}{18pt}
\pagestyle{fancy}
\fancyhf{}
\fancyhead[L]{\includegraphics[height=5mm]{./figures/logo}}
\fancyfoot[R]{\thepage}

% date format
\renewcommand{\datename}{\emph{Fecha:}}


% ------------------ %

% Theorem environments

% colors for theorem environments
\definecolor{myqsbg}{HTML}{f2fbfc}
\definecolor{myqsfr}{HTML}{191971}

% problem environment
\tcbuselibrary{theorems,skins,hooks}
\newtcbtheorem{question}{Problema}
{
	enhanced,
	breakable,
	colback = myqsbg,
	frame hidden,
	boxrule = 0sp,
	borderline west = {2pt}{0pt}{myqsfr},
	sharp corners,
	detach title,
	before upper = \tcbtitle\par\smallskip,
	coltitle = myqsfr,
	fonttitle = \bfseries\sffamily,
	description font = \mdseries,
	separator sign none,
	segmentation style={solid, myqsfr},
}
{th}

% Environment for problems
\newtheorem{problem}{Problema}

% simplified theorem environment

\newcommand{\qs}[2]{\begin{question}{#1}{}#2\end{question}}

% solution command
\newcommand{\sol}[1]{\noindent\textbf{\textit{Solución:}} #1 \par \hfill $_\square$ }
\newcommand{\dem}[1]{\noindent\textbf{\textit{Demostración:}} #1 \par \hfill $_\square$ }

% shortcuts for some math symbols
\newcommand{\R}{\mathbb{R}}
\newcommand{\N}{\mathbb{N}}
\newcommand{\Z}{\mathbb{Z}}
\newcommand{\Q}{\mathbb{Q}}
\newcommand{\C}{\mathbb{C}}

\newcommand{\Bb}{\mathcal{B}}
\newcommand{\ps}[1]{\mathcal{P}\left(#1\right)}

\newcommand{\dd}{\mathrm{d}}
\newcommand{\ZnZ}{\Z / n\Z}

% shortcuts for some math operators
\newcommand{\abs}[1]{\left\lvert #1 \right\rvert}
\newcommand{\norm}[1]{\left\lVert #1 \right\rVert}

\newcommand{\ol}[1]{\overline{#1}}
\usepackage{fontawesome5}
\usetikzlibrary{decorations.pathreplacing,calligraphy}
\title{Cuadratura real - Historia de las matemáticas}
\author{Jose Prado, Diego Tarapuez, Kevin Velez}
\date{Octubre 2022}

\begin{document}
\maketitle \thispagestyle{fancy}
\qs{Cuadratura real de un rectángulo}{
    Dado un rectángulo cualquiera $a \cdot b$, como convertirlo en un cuadrado $c^2$, de tal manera que $a \cdot b = c^2$. usando solo recortes, traslaciones y rotaciones.
}

\dem{
    Sea el rectángulo dado $a \cdot b$ de la figura \ref{fig:rectangulo_original}

    \begin{figure}[H]
        \centering
        \begin{tikzpicture}
            \draw[thick] (0,0) rectangle (6,3);
            \node[left] at (0, 1.5) {$a$};
            \node[below] at (3, 0) {$b$};
        \end{tikzpicture}
        \label{fig:rectangulo_original}
        \caption{Rectángulo dado $a \cdot b$}
    \end{figure}

    Sobreponemos el lado más pequeño del rectángulo un segmento de la longitud del lado del cuadrado, es decir, $c$, de tal manera que $c^2 = a \cdot b$.

    Luego, trazamos un corte, siguiendo la recta que una al vértice inferior derecho del rectángulo con el extremo superior del segmento trazado.

    \begin{figure}[H]
        \begin{tikzpicture}

            \fill[cyan!10] (6,0) -- (6,3) -- +(-{18^(1/2)},0) -- cycle;

            \draw[thick] (0,0) rectangle (6,3);
            
            \node[below] at (3, 0) {$b$};

            \draw[blue, thick] (0,0) -- (0,{18^(1/2)});

            \draw[dashed] (6,0) -- (0,{18^(1/2)});
            \draw(6,0) ++ (144.74:1) node[rotate=144.74] {\faCut};

            \draw[|{Latex}-{Latex}|] (-0.25,0) -- (-0.25,3);
            \node[fill=white] at (-0.25, 1.5) {$a$};
            \draw[|{Latex}-{Latex}|] (-0.75,0) -- (-0.75,{18^(1/2)});
            \node[fill=white] at (-0.75, {18^(1/2)/2}) {$c$};

            \draw[|{Latex}-{Latex}|] ({6-18^(1/2)},3.3) --+({18^(1/2)},0);
            \node[fill=white] at ({6-18^(1/2)/2},3.3) {$x$};
        \end{tikzpicture}
    \end{figure}

    Después, desplazamos el triángulo recortado, hasta que el vértice superior izquierdo coincida con el extremo del segmento $c$ trazado inicialmente.
    
    \newpage

    \begin{figure}[H]
        \centering
        \begin{tikzpicture}
            \fill[cyan!10] ({18^(1/2)}, {18^(1/2)}) -- +(0,-3) -- (0,{18^(1/2)}) -- cycle;

            %\draw[thick] (0,0) -- (0,{18^(1/2)});
            \draw[thick] (0,0) -- (0,3);
            \draw[thick] (6,0) -- (0,{18^(1/2)});
            \draw[thick] (0,3) -- + ({6-18^(1/2)}, 0);

            \draw[thick] (0,0) -- (6,0);

            \draw[thick] (0,{18^(1/2)}) -- ({18^(1/2)},{18^(1/2)});
            \draw[thick] ({18^(1/2)}, {18^(1/2)}) -- +(0,-3);

            \draw[thick, blue] (0,0) -- (0,{18^(1/2)});
            \node[left, thick] at (0, {18^(1/2)/2}) {$c$};
        \end{tikzpicture}
    \end{figure}
    
    Seguido a esto, recortamos el triángulo sobrante, siguiendo la recta del triángulo desplazado.
    
    \begin{figure}[H]
        \centering
        \begin{tikzpicture}
            \fill[cyan!10] ({18^(1/2)}, {18^(1/2)}) -- +(0,-3) -- (0,{18^(1/2)}) -- cycle;
            \fill[violet!15] (6,0) -- ({18^(1/2)},0) -- ({18^(1/2)},{18^(1/2)-3}) -- cycle;

            %\draw[thick] (0,0) -- (0,{18^(1/2)});
            \draw[thick] (0,0) -- (0,3);
            \draw[thick] (6,0) -- (0,{18^(1/2)});
            \draw[thick] (0,3) -- + ({6-18^(1/2)}, 0);

            \draw[thick] (0,0) -- (6,0);

            \draw[thick] (0,{18^(1/2)}) -- ({18^(1/2)},{18^(1/2)});
            \draw[thick] ({18^(1/2)}, {18^(1/2)}) -- +(0,-3);

            \draw[thick, blue] (0,0) -- (0,{18^(1/2)});
            \node[left, thick] at (0, {18^(1/2)/2}) {$c$};

            \draw[dashed] ({18^(1/2)},0) -- + (0, {18^(1/2)-3});
            \node[rotate=90] at ({18^(1/2)}, {(18^(1/2)-3)/4}) {\faCut};
        \end{tikzpicture}
    \end{figure}

    Por último, desplazamos el triángulo recortado, al espacio restante que no queda, para así formar finalmente el cuadrado.

    \begin{figure}[H]
        \centering
        \begin{tikzpicture}
            \fill[cyan!10] ({18^(1/2)}, {18^(1/2)}) -- +(0,-3) -- (0,{18^(1/2)}) -- cycle;
            \fill[violet!15] (0,{18^(1/2)}) -- (0,3) -- +({18^(1/2)-2.5}, 0);
            \draw[thick] (0,0) -- (0,3);
            \draw[thick] (0,3) -- + ({6-18^(1/2)}, 0);
            \draw[thick] (0,{18^(1/2)}) -- ({18^(1/2)}, {18^(1/2)-3});
            \draw[thick] (0,0) -- ({18^(1/2)},0);
            \draw[thick] ({18^(1/2)},0) -- +(0, {18^(1/2)-3});

            \draw[thick] (0,{18^(1/2)}) -- ({18^(1/2)},{18^(1/2)});
            \draw[thick] ({18^(1/2)}, {18^(1/2)}) -- +(0,-3);

            \draw[thick, blue] (0,0) -- (0,{18^(1/2)});
            \node[left, thick] at (0, {18^(1/2)/2}) {$c$};
        \end{tikzpicture}
    \end{figure}

    \textit{\textbf{Nota 1:}} El último triángulo recortado (el sobrante) y el triángulo en el espacio que queda al desplazar el primer triángulo son congruentes, por lo que la construcción tiene sentido. Esto debido a que ambos son triángulos rectángulos, por lo que tienen un ángulo igual. Además, su hipotenusa es la misma recta, que es transversal a dos paralelas, por lo que sus ángulos correspondientes son iguales, y en consecuencia sus tres ángulos son iguales.

    Por otro lado, el triángulo al desplazarse hacia arriba, deja la misma magnitud por debajo de la que recorre hacia arriba, por lo que los dos triángulos de interés tienen un lado igual, luego, por congruencia ALA los dos triángulos son congruentes.

    \textit{\textbf{Nota 2:}} La figura obtenida es, en efecto, un cuadrado, ya que, si nos fijamos en la segunda figura, los triángulos que se forman son semejantes (el de base $b$ y altura $c$ y el de base $x$ y altura $a$). Por lo que, de la semejanza se tiene que 
    $$ \frac{c}{b} = \frac{a}{x}$$
    De donde, $x = \frac{ab}{c}$, y como $ab=c^2$, entonces $x=c$.
}

\newpage

\qs{}{
    Dado el arreglo de la siguiente figura
    \begin{center}
        \begin{tikzpicture}
            \filldraw (0,0) circle (1.5pt);
            \filldraw (6,0) circle (1.5pt);
            \draw[thick] (0,0) -- (6,0);

            \filldraw (3,0) circle (1.5pt);
            \filldraw (4,0) circle (1.5pt);
            
            \draw [thick, violet, decorate, decoration = {brace}] (0,0.2) --  (3,0.2);
            \draw [thick, violet, decorate, decoration = {brace}] (3,0.2) --  (6,0.2);

            \draw [thick, magenta, decorate, decoration = {brace}, xshift=4cm, rotate=180] (0,0.2) --  (4,0.2);
            \draw [thick, magenta, decorate, decoration = {brace}, xshift=10cm, rotate=180] (4,0.2) --  (6,0.2);

            \node[violet, above] at (1.5,0.3) {$a$};
            \node[violet, above] at (4.5,0.3) {$a$};

            \node[magenta, below] at (2,-0.3) {$b$};
            \node[magenta, below] at (5,-0.3) {$c$};

            \draw [thick, olive, decorate, decoration = {brace}, xshift=7cm, rotate=180] (3,0.4) --  (4,0.4);

            \node[olive, below] at (3.5,-0.5) {$d$};
        \end{tikzpicture}
    \end{center}
    Mostrar que $a^2 = b \cdot c + d^2$
}

\dem{
    Construimos un cuadrado de lado $a$ y un rectángulo de lados $b$ y $c$, con un lado del rectángulo sobre un lado del cuadrado.

    \begin{center}
        \begin{tikzpicture}
            \fill[fill=olive!30, thick] (3,3) -- (3,4) -- (4,4) -- (4,3) -- cycle;

            \draw[thick] (0,0) -- (4,0) -- (4,4) -- (0,4) -- cycle;
            \draw[thick] (0,0) -- (5,0) -- (5,3) -- (0,3) -- cycle;
            
            \draw[|{Latex}-{Latex}|] (-0.3, 0) -- (-0.3, 3);
            \node[fill=white] at (-0.3, 1.5) {$c$};
            
            \draw[|{Latex}-{Latex}|] (-0.6,0) -- (-0.6,4);
            \node[fill=white] at (-0.6, 2) {$a$};
            
            \draw[|{Latex}-{Latex}|] (0,-0.3) -- (5,-0.3);
            \node[fill=white] at (2.5, -0.3) {$b$};

            \draw[|{Latex}-{Latex}|] (5.3,3) -- (5.3,4);
            \node[fill=white] at (5.3, 3.5) {$d$};

            \draw[|{Latex}-{Latex}|] (4,4.3) -- (5,4.3);
            \node[fill=white] at (4.5, 4.3) {$d$};

            \draw[dashed] (3,0) -- (3,4);
            \draw[|{Latex}-{Latex}|] (0,-0.6) -- (3,-0.6);
            \node[fill=white] at (1.5, -0.6) {$c$};
            
            \filldraw[draw=black, fill=violet!15] (4,0) -- (5,0) -- (5,3) -- (4,3) -- cycle;

            \filldraw[draw=black, fill=violet!15] (0,3) -- (0,4) -- (3,4) -- (3,3) -- cycle;

            \draw[blue, thick] (0,0) rectangle (5,3);
            \draw[red, thick] (0,0) rectangle (4,4);

            \node at (4.5,1.5) {$c \cdot d$};
            \node at (1.5, 3.5) {$c \cdot d$};
            
            \node at (3.5, 3.5) {$d^2$};            

        \end{tikzpicture}
    \end{center}

    Observemos que los dos rectángulos coloreados son iguales, el rectángulo $c \cdot d$ que sobra del rectángulo $b \cdot c$ al cuadrado $a^2$ es el mismo que el rectángulo que sobra del cuadrado $a^2$ al rectángulo $b \cdot c$ menos el cuadrado $d^2$. Por lo que $$a^2 = b \cdot c + d^2$$
}



\end{document}