\documentclass{amsart}
% Language and encoding
\usepackage[spanish]{babel}
\usepackage[utf8]{inputenc}

% margins
% \usepackage[lmargin=2.5cm, rmargin=2cm]{geometry}
\usepackage[lmargin=3cm, rmargin=2cm, bmargin=2cm, tmargin=2cm]{geometry}

% Math packages
\usepackage{amsmath}
\usepackage{amssymb}
\usepackage{amsthm}
\usepackage{mathrsfs}

% graphics and color
\usepackage{graphicx}
\usepackage{xcolor}
\usepackage{float}
\usepackage{subfigure}
\usepackage{wrapfig}

% color boxes 
\usepackage[most,many,breakable]{tcolorbox}

% fancy headers
\usepackage{fancyhdr}

% Tikz
\usepackage{tikz}
\usetikzlibrary{angles, calc, arrows, arrows.meta}
\AtBeginEnvironment{tikzpicture}{\shorthandoff{>}\shorthandoff{<}}{}{}

% cancel
\usepackage[makeroom]{cancel}
\newcommand\cancelc[2][black]{\renewcommand\CancelColor{\color{#1}}\cancel{#2}}

% fancy settings
\setlength{\headheight}{18pt}
\pagestyle{fancy}
\fancyhf{}
\fancyhead[L]{\includegraphics[height=5mm]{./figures/logo}}
\fancyfoot[R]{\thepage}

% date format
\renewcommand{\datename}{\emph{Fecha:}}


% ------------------ %

% Theorem environments

% colors for theorem environments
\definecolor{myqsbg}{HTML}{f2fbfc}
\definecolor{myqsfr}{HTML}{191971}

% problem environment
\tcbuselibrary{theorems,skins,hooks}
\newtcbtheorem{question}{Problema}
{
	enhanced,
	breakable,
	colback = myqsbg,
	frame hidden,
	boxrule = 0sp,
	borderline west = {2pt}{0pt}{myqsfr},
	sharp corners,
	detach title,
	before upper = \tcbtitle\par\smallskip,
	coltitle = myqsfr,
	fonttitle = \bfseries\sffamily,
	description font = \mdseries,
	separator sign none,
	segmentation style={solid, myqsfr},
}
{th}

% simplified theorem environment
\newcommand{\qs}[2]{\begin{question}{#1}{}#2\end{question}}

% Environment for problems
\newtheorem{problem}{Problema}

% solution command
\newcommand{\sol}[1]{\noindent\textbf{\textit{Solución:}} #1 \par \hfill $_\square$ }
\newcommand{\dem}[1]{\noindent\textbf{\textit{Demostración:}} #1 \par \hfill $_\square$ }

% shortcuts for some math symbols
\newcommand{\R}{\mathbb{R}}
\newcommand{\N}{\mathbb{N}}
\newcommand{\Z}{\mathbb{Z}}
\newcommand{\Q}{\mathbb{Q}}
\newcommand{\C}{\mathbb{C}}

\newcommand{\Bb}{\mathcal{B}}
\newcommand{\ps}[1]{\mathcal{P}\left(#1\right)}

\newcommand{\dd}{\mathrm{d}}
\newcommand{\ZnZ}{\Z / n\Z}

% shortcuts for some math operators
\newcommand{\abs}[1]{\left\lvert #1 \right\rvert}
\newcommand{\norm}[1]{\left\lVert #1 \right\rVert}

\newcommand{\ol}[1]{\overline{#1}}

\title{TAREA 3}
\author{Alejandro Umaña, Kevin Velez}
\date{Octubre 2022}

\begin{document}
\maketitle \thispagestyle{fancy}

\qs{}{
    Sean $A$, $B$, segmentos, demuestre que si $ \frac{A}{B} = \frac{C}{D} $ entonces $ \frac{A}{C} = \frac{B}{D} $
}

\dem{
    Por hipotesis tenemos que $ \frac{A}{B} = \frac{C}{D} $ esto significa por definición que si $ nA = mB $ entonces $nC = mD$

    $ \Rightarrow)$ Supongamos que $ \frac{A}{C}$ esto significa que $nA = nC$, y por tanto tenemos $nA = nC = mB$

    Partiendo de $nA = mB$ agregamos cosas iguales a cosas iguales

    \begin{align*}
        nA + nC &= nC + mB \quad \mbox{Pero} \quad nC = mD \\
        nA + mD &= nC + mB \quad \mbox{Pero} \quad nC = nA \\
        nA + mD &= mB + nA \quad \mbox{Quitando coas iguales} \\
        mD &= mB
    \end{align*}

    Esto quiere decir que si $nA = nC$ entonces $mD = mB$ pero por definición significa que $\frac{A}{C} = \frac{B}{D}$.

    $\Leftarrow)$ Supongamos que $\frac{B}{D}$ esto significa por definición que $mB = mD$, por tanto $mD = mB = nC$

    Partiendo de $nA = mB$ entonces $nA + mD = mB + mD$, pero $mD = nC$, por lo que $nA + mD = mB + nC$, pero $mD = mB$ entonces $nA + mB = mB + nC$, así que $nA = nC$.
    Esto significa que $\frac{A}{C}$, por tanto $nA = mB$, entonces $nA = nC \Rightarrow \frac{B}{D} = \frac{A}{C}$.
}

\qs{}{
    Sean $A$, $B$ segmentos, con $A>B$. Demuestre que $ \frac{A}{G} > \frac{B}{G} $
}

\dem{
    Supongamos que $A > B$. Sea $C$ el segmento tal que $B + C  = A$.
    
    Tomemos $mA, mB$ y $nG$ tal que $mA, mC > G$. Tomemos $nG$ el menor múltiplo de $G$ que hace que $nG \geq mB$. Por tanto, $mB$ no es mayor que $nG$.

    Pero $mA > nG$, en efecto, como $nG$ es el menor múltiplo que hace $nG \geq mB$, entonces $nG - G < mB$ y tenemos que $G < mC$, por tanto, $nG - \cancelc[red]{G} + \cancelc[red]{G} < mB + mC$, es decir $nG < mA$.

    Luego, tenemos que $\frac{A}{G} > \frac{B}{G}$.
}

\end{document}