\documentclass{amsart}
% Language and encoding
\usepackage[spanish]{babel}
\usepackage[utf8]{inputenc}

% margins
% \usepackage[lmargin=2.5cm, rmargin=2cm]{geometry}
\usepackage[lmargin=3cm, rmargin=2cm, bmargin=2cm, tmargin=2cm]{geometry}

% Math packages
\usepackage{amsmath}
\usepackage{amssymb}
\usepackage{amsthm}

% graphics and color
\usepackage{graphicx}
\usepackage{xcolor}
\usepackage{float}
\usepackage{subfigure}
\usepackage{wrapfig}

% color boxes 
\usepackage[most,many,breakable]{tcolorbox}

% fancy headers
\usepackage{fancyhdr}

% Tikz
\usepackage{tikz}
\usetikzlibrary{angles, calc, arrows, arrows.meta}
\AtBeginEnvironment{tikzpicture}{\shorthandoff{>}\shorthandoff{<}}{}{}

% cancel
\usepackage[makeroom]{cancel}
\newcommand\cancelc[2][black]{\renewcommand\CancelColor{\color{#1}}\cancel{#2}}

% fancy settings
\setlength{\headheight}{18pt}
\pagestyle{fancy}
\fancyhf{}
\fancyhead[L]{\includegraphics[height=5mm]{./figures/logo}}
\fancyfoot[R]{\thepage}

% date format
\renewcommand{\datename}{\emph{Fecha:}}


% ------------------ %

% Theorem environments

% colors for theorem environments
\definecolor{myqsbg}{HTML}{f2fbfc}
\definecolor{myqsfr}{HTML}{191971}

% problem environment
\tcbuselibrary{theorems,skins,hooks}
\newtcbtheorem{question}{Problema}
{
	enhanced,
	breakable,
	colback = myqsbg,
	frame hidden,
	boxrule = 0sp,
	borderline west = {2pt}{0pt}{myqsfr},
	sharp corners,
	detach title,
	before upper = \tcbtitle\par\smallskip,
	coltitle = myqsfr,
	fonttitle = \bfseries\sffamily,
	description font = \mdseries,
	separator sign none,
	segmentation style={solid, myqsfr},
}
{th}

% Environment for problems
\newtheorem{problem}{Problema}

% simplified theorem environment

\newcommand{\qs}[2]{\begin{question}{#1}{}#2\end{question}}

% solution command
\newcommand{\sol}[1]{\noindent\textbf{\textit{Solución:}} #1 \par \hfill $_\square$ }
\newcommand{\dem}[1]{\noindent\textbf{\textit{Demostración:}} #1 \par \hfill $_\square$ }

% shortcuts for some math symbols
\newcommand{\R}{\mathbb{R}}
\newcommand{\N}{\mathbb{N}}
\newcommand{\Z}{\mathbb{Z}}
\newcommand{\Q}{\mathbb{Q}}
\newcommand{\C}{\mathbb{C}}

\newcommand{\Bb}{\mathcal{B}}
\newcommand{\ps}[1]{\mathcal{P}\left(#1\right)}

\newcommand{\dd}{\mathrm{d}}
\newcommand{\ZnZ}{\Z / n\Z}

% shortcuts for some math operators
\newcommand{\abs}[1]{\left\lvert #1 \right\rvert}
\newcommand{\norm}[1]{\left\lVert #1 \right\rVert}

\newcommand{\ol}[1]{\overline{#1}}

\title{Tarea 1}
\author{Kevin Velez}
\date{Octurbe, 2022}

\begin{document}
\maketitle \thispagestyle{fancy}

\section*{7.1 BASIC DEFINITIONS AND EXAMPLES}

\begin{problem}[13]
    An element $x$ in $R$ is called \emph{nilpotent} if $x^m = 0$ fo some $m \in \Z^+$

    \begin{enumerate}
        \item Show that if $ n = a^k b $ form some integers $a$ and $b$ then $ab$ is a nilpotent element of $ \Z / n\Z $.
        \item If $ a \in \Z $ is an integer, show that the element $ a \in \Z / n\Z $ is nilpotent if and only if  every prime divisor of $n$ is also a divisor of $a$. In particular, determine the nilpotent  elements of $ \Z / 72\Z $ explicitly.
        \item Let $R$ be the ring of functions from a nonempty set $X$ to a field $F$. Prove that $R$  contains no nonzero nilpotent elements. 
    \end{enumerate}
\end{problem}

\begin{proof} \,
    \begin{enumerate}
        \item Let $n = a^k b$ for some integers $a$ and $b$.
        Then $$ (ab)^k = a^k b^k = (a^k b) b^{k-1} = n b^{k-1} \equiv 0 \mod{n} $$
        
        \item Suppose $ \ol{a} \in \ZnZ $ is nilpotent. Then $ \ol{a}^k = \ol{0} $ for some $ k \in \Z^+ $. Then $a^k \equiv 0 \mod{n}$, so $n | a^k$. Now, if $p$ is a prime divisor of $n$, then $p | a^k$ and therefore $p | a$.
        
        Now, Suppose that every prime divisor of $n$ is a divisor of $a$. let $ n = {p_1}^{\alpha_1} \cdots {p_k}^{\alpha_k} $ and $ a = {p_1}^{\beta_1} \cdots {p_k}^{\beta_k} m$ where $1 \leq \alpha_i, \beta_i$ for all $i$ and for some integer $m$. Let $s = \max{\alpha_i}$. Then
        $$ a^s = \left( {p_1}^{\beta_1} \cdots {p_k}^{\beta_k} m \right)^s = {p_1}^{\beta_1 s} \cdots {p_k}^{\beta_k s} m^s$$
        where $\beta_i s \geq \alpha i$, thus
        $$ a^s = n {p_1}^{\beta_1 s - \alpha_1} \cdot {p_k}^{\beta_k s - \alpha_k} m$$
        And therefore $ a \equiv 0 \mod{n} $
        
        $ 72 = 2^3 3 ^2 $, then the nilpotent elements of $ \Z / 72\Z $ are  $\{ 0, 6, 12, 18, 24, 30, 36, 42, 48, 54, 60, 66 \}$.
        
        \item Suppose $ \alpha \in R $ is nilpotent. If $\alpha \neq 0$, then exists $ x \in X $ such that $ \alpha(x) \neq 0 $. Let $m$ be the smallest integer such that $ \alpha(x)^m = 0 $; note that $ m \geq 1 $. Then $ \alpha(x) {\alpha(x)}^{m-1} $, where $ \alpha(x) $ and $ {\alpha(x)}^{m-1} $ are not zero. Thus $F$ contains zero divisor, which is a contradiction. Thus $R$ contains no nonzero nilpotent elements.
    \end{enumerate}
\end{proof}

\begin{problem}[21]
    Let $X$ e any nonempty set and let $\ps{X}$ the set of all subsets of $X$ (the power set of $X$). Define addition and multiplication on $\ps{X}$ by
    $$ A + B = (A-B) \cup (B-A) \qquad \mbox{and} \qquad A \times B =  A \cap B $$
    i.e. addition is the symmetric difference and multiplication is the intersection.

    \begin{enumerate}
        \item Prove that $\ps{X}$ is a ring under these operations ($\ps{X}$ and its subrings are often referred to as rings of sets).
        \item Prove that this ring is commutative, has a identity and is a Boolean ring.
    \end{enumerate}
\end{problem}

\begin{proof} \, 
    \begin{enumerate}
        \item Sean $A, B, C \in \ps{X}$, entonces
        \begin{align*}
            &\ (A + B) + C\\ =&\ ((A - B) \cup (B - A)) + C\\
            =&\ (((A - B) \cup (B - A)) - C) \cup (C - ((A - B) \cup (B - A)))\\
            =&\  ((A - B) - C) \cup ((B - A) - C) \cup ((C -(A - B)) \cap (C - (B - A)))\\
            =&\ (A - (B \cup C)) \cup (B - (A \cup C)) \cup (((C - A) \cup (C \cap B)) \cap ((C - B) \cup (C \cap A)))\\
            =&\  [(A - B) \cap (A - C)] \cup [(B - A) \cap (B - C)] \cup [(C - A) \cap (C - B)] \\ 
            &\ \qquad \qquad \qquad \qquad \cup [(C - A) \cap C \cap A] \cup [(C - B) \cap C \cap B] \cup [C \cap B \cap C \cap A]\\
            =&\  [(A - B) \cap (A - C)] \cup [(B - A) \cap (B - C)] \cup [(C - A) \cap (C - B)] \cup [A \cap B \cap C]\\
            =&\  [(A - B) \cap (A - C)] \cup [(B - A) \cap (B - C)] \cup [(C - A) \cap (C - B)] \\ 
            &\ \qquad \qquad \qquad \qquad \cup [(A - B) \cap (A \cap B)] \cup [(A - C) \cap (A \cap C)] \cup [A \cap B \cap A \cap C]\\
            =&\ [((A - B) \cup (A \cap C)) \cap ((A - C) \cup (A \cap B)] \cup [(B - A) \cap (B - C)] \cup [(C - A) \cap (C - B)]\\
            =&\ [(A - (B - C)) \cap (A - (C - B))] \cup [((B - A) \cap (B - C)) \cup ((C - A) \cap (C - B))]\\
            =&\ (A - ((B - C) \cup (C - B))) \cup [(B - (C \cup A)) \cup (C - (B \cup A))]\\
            =&\ (A - (B+C)) \cup [((B - C) - A) \cup ((C - B) - A)]\\
            =&\ (A - (B + C)) \cup (((B - C) \cup (C - B)) - A)\\
            =&\ (A - (B+C)) \cup ((B+C)- A)\\
            =&\  A + (B+C)
        \end{align*}
        Entonces $+$ es asociativa.

        Veamos que
        $$ A + \emptyset = (A - \emptyset) \cup (\emptyset - A) =  A \cup \emptyset = A $$
        Y similarmente $\emptyset + A = A$, por lo que $\emptyset$ es el elemento neutro de $+$.
        
        Veamos ahora que
        $$ A + A = (A \setminus A) \cup (A \setminus A) = \emptyset \cup \emptyset = \emptyset$$
        Por lo que, cada elemento de $R$ es su propio inverso aditivo. Por tanto $(R,+)$ es un grupo.

        Además 
        $$ A + B = (A \setminus B) \cup (B \setminus A) = (B \setminus A) \cup (A \setminus B) = B + A $$
        Por lo que $(R, +)$ es un grupo abeliano.

        $$ A \cdot (B \cdot C) = A \cdot (B \cap C ) = A \cap (B \cap C) = (A \cap B) \cap C = (A \cdot B) \cdot C $$
        Así que la multiplication es asociativa.

        \begin{align*}
            A \cdot (B + C) =&\ A \cap ((B \setminus C) \cup (C \setminus B))\\
            =&\ (A \cap (B \setminus C)) \cup (A \cap (C \setminus B))\\
            =&\ ((A \cap B) \setminus (A \cap C)) \cup ((A \cap B) \setminus (A \cap C))\\
            =&\ (A \cap B) + (A \cap C) \\
            =&\ A \cdot B + A \cdot C;
        \end{align*}
        Así que la multiplication es distributiva sobre la addition. (La prueba por la derecha es similar.) ya que $\cap$ es conmutativo. Por lo tanto, $R$ es un anillo.
        
        \item Como $A\cdot B = A \cap B = B \cap A = B \cdot A$, entonces $R$ es conmutativo.
        
        Como $A \cdot X = A \cap C = A$ y $X \cdot A = X \cap A = A$, entonces $R$ tiene identidad.

        Además, para todo $A \in R$, $A \cdot A = A \cap A = A$, por lo que $R$ es booleano.
    \end{enumerate}
\end{proof}

\begin{problem}[25]
    Sea $I$ el anillo de los cuaterniones de Hamilton, y definimos
    $$ N : I \to \Z \quad \mbox{by} \quad N(a + bi +cj + dk) = a^2 + b^2 + c^2 + d^2$$
    (La función $N$ es llamada \emph{Norma}).

    \begin{enumerate}
        \item Pruebe that $ N(\alpha) = \alpha \ol{\alpha} $ para todo $\alpha \in I$, donde si $\alpha = a + bi + cj + dk$, entonces $\ol{\alpha} = a - bi - cj - dk$.
        \item Pruebe que $N(\alpha\beta) = N(\alpha)N(\beta)$ para todo $\alpha, \beta \in I$.
        \item Pruebe que un elemento de $I$ es una unidad si y solo si su norma $N(\alpha) = +1$. Muestre que $I^*$ es isomorfo al grupo de cuaterniones de orden 8 [El inverso en el anillo de los cuaterniones racionales de un elemento diferente de cero $\alpha$ es $\frac{\ol{\alpha}}{N(\alpha)}$.]
    \end{enumerate}
\end{problem}
\begin{proof}
    \,
    \begin{enumerate}
        \item \, 
        \begin{align*}
            \alpha \overline{\alpha} &= (a+bi+cj+dk)(a-bi-cj-dk) \\
            &= a^2 - abi - acj - adk + bai - b^2i^2 - bcij - bdik + caj \\
            &\ \qquad \qquad - cbji - c^2j^2 - cdjk + dak - daki - dckj - d^2k^2 \\
            &= a^2+b^2+c^2+d^2 \\ 
            &= N(\alpha)
        \end{align*}
        \item \,
        \begin{align*}
            N(\alpha\beta) &= N((a+bi+cj+dk)(x+yi+zj+wk)) \\
            &= N((ax-by-cz-dw) +(ay+bx+cw-dz)i + \\
            &\ \qquad \qquad (az-bw + cx+dy)j +(aw+bz-cy+dx)k) \\
            &= (ax-by-cz-dw)^2 + (ay+bx+cw-dz)^2 + \\
            &\ \qquad \qquad (az-bw+cx+dy)^2  +(aw+bz-cy+dx)^2\\ 
            &= (a^2 + b^2 + c^2 + d^2)(x^2 + y^2 + z^2 + w^2)\\ 
            &= N(\alpha)N(\beta)
        \end{align*}
        
        \item Supongamos que $\alpha$ es una unidad, entonces $\alpha\beta = 1$ para algún $\beta$ en el anillo  de cuaterniones de Hamilton. Note que por la definición de $N$ como una suma de cuadrados, entonces $N(\alpha) \geq 0$ para todo $\alpha \in I$. Ahora
        $$ 1 = N(1) = N(\alpha\beta) = N(\alpha)N(\beta) $$
        y $N(\alpha)$ y $N(\beta)$ son ambos enteros, por lo tanto $N(\alpha) = 1$. Ahora, supongamos que $N(\alpha) = 1$. Entonces
        $$ \alpha \ol{\alpha} = N(\alpha) = 1$$
        Y claramente $\ol{\alpha} \in I$, así que $\alpha$ es una unidad en $I$.

        Supongamos que $\alpha \in I$ es una unidad, entonces
        $$ N(\alpha) = a^2 + b^2 + c^2 + d^3 = 1$$
        si alguno de $a, b, c, d$ es mayor que 2 en valor absoluto, tenemos una contradicción, así que cada uno de $a, b, c, d$ es menor que 2 en valor absoluto. Por lo tanto son $1, 0$ o  $-1$, Si más de uno es mayor que 1 en valor absoluto, tenemos otra contradicción; por lo tanto, a lo mucho uno de ellos puede set $1$, si todos son cero, entonces $\alpha = 0$, por lo que no es una unidad, así que exactamente uno de ellos es 1, entonces $N(\alpha) = 1$, así que $\alpha$ es una unidad. Por lo tanto $\abs{I^*} = 8$, ya que $I^*$ es no abeliano y tiene seis elementos de orden 4, $I^* \cong Q_8$.
    \end{enumerate}
\end{proof}

\begin{problem}[30]
    Sea $A = \Z \times \Z \times \Z \times \cdots $ el producto de contables copias de $\Z$. Sea $A$ el anillo con adición y multiplicación componente a componente. Sea $R$ el anillo de los homomorfismos de  grupo de $A$ en si mismo descrito en el ejercicio anterior. Es decir, con la adición y composición punto por punto. Sea $\varphi$ el elemento de $R$ definido por $\varphi(a_1,a_2,a_3,\ldots) = (a_2,a_3,a_4,\ldots)$. Sea $\psi$ el elemento de $R$ de definido por $\psi(a_1,a_2,a_3,\ldots) = (0,a_1,a_2,a_3,\ldots)$.
    \begin{enumerate}
        \item Probar que $\varphi\psi$ es la identidad de $R$ pero $\psi\varphi$ no es la identidad de $R$.
        \item exhibir infinitos inversor por derecha de $\varphi$ en $R$.
        \item Encontrar un elemento no cero $\pi$ en $R$ tal que $\varphi\pi = 0$ pero $\pi\varphi \neq 0$.
        \item Probar que no hay un elemento distinto de cero $\lambda \in R$ tal que $\lambda\varphi = 0$ 
    \end{enumerate}
\end{problem}

\begin{proof}
    \, 
    \begin{enumerate}
        \item Sea $\left( \prod a_i \right) \in A$. Note que 
        $$ (\varphi \circ \psi) \left( \prod a_i\right) = \varphi \left(\prod b_i\right)$$
        Donde $b_0 = 0$ y $b_{i+1} = a_i$. Ahora, $\varphi \left(\prod  b_i\right) = \prod c_i$ donde $c_i = b_{i+1} = a_i$ para todo $i \in \N$; por lo tanto $(\varphi \circ \psi) \left(\prod a_i\right)$ y tenemos que $\varphi \circ \psi = 1$. Por otro lado, si $a_0 \neq 0$, entonces la cero-esima coordinada de $\left(a_i\right)$ es no cero, mientras que la cero-esima coordinada de $(\psi\circ\vartriangleleft)\left(\prod a_i\right)$ es cero, por lo que $\psi \circ \varphi \neq 1$
        
        
    \end{enumerate}
\end{proof}

\end{document}