\begin{problem}[21] Sea X cualquier subconjunto no vacío y sea $\mathcal{P} (X)$ el conjunto de todos los subconjuntos de $X$. Se define la suma y multiplicación sobre $\mathcal{P} (X)$ por \\
\begin{center}
    $A+B = (A - B) \cup (B - A) $ \hspace{3mm} y \hspace{3mm} $A\times B = A \cap B$
\end{center}
\vspace{3mm}

es decir, la suma es la diferencia simétrica de conjuntos y el producto es la intersección.\\

a) Pruebe que $\mathcal{P}(X)$ es un anillo bajo estas operaciones.\\
b) Pruebe que este anillo es conmutativo, tiene identidad y es booleano.\\


\end{problem}

\begin{proof}
a) $(\mathcal{P}(X), + )$ es un grupo abeliano.\\

\textit{i)} Sean $A,B,C \in \mathcal{P}(X)$, $(A+B)+C = A+(B+C)$ puesto que la diferencia simétrica es asociativa. \\

\textit{ii)} Sea $A \in \mathcal{P}(X)$, 

\begin{equation*}
\begin{split}
    A + \emptyset &= (A - \emptyset) \cup (\emptyset - A)\\
                  &= A \cup \emptyset\\
                  &= A
\end{split}
\end{equation*}

Análogamente se tiene que $\emptyset + A = A$. Por tanto $\emptyset$ es el elemento neutro de +.\\

\textit{iii)} Sea $A \in \mathcal{P}(X)$, 
\begin{equation*}
\begin{split}
    A+A &= (A-A) \cup (A-A)\\
        &= \emptyset \cup \emptyset\\
        &= \emptyset
\end{split}
\end{equation*}

esto es, $A$ es su propio inverso.\\

\textit{iv)} Sean $A,B \in \mathcal{P}(X)$,

\begin{equation*}
\begin{split}
    A+B &= (A-B) \cup (B-A)\\
        &= (B-A) \cup (A-B)\\
        &= B + A
\end{split}
\end{equation*}

\vspace{3mm}

$\times$ es asociativo.\\

\begin{equation*}
\begin{split}
    (A\times B)\times C &= (A \cap B)\times C\\
                      &= (A \cap B)\cap C \\
                      &= A \cap (B \cap C)\\
                      &= A \times (B \cap C)\\
                      &= A \times (B \times C)
\end{split}
\end{equation*}

\vspace{3mm}

Valen las propiedades distributivas para $A,B,C \in \mathcal{P}(X)$

\begin{equation*}
\begin{split}
    A\times (B+C) &= A\times( (B-C) \cup (C-B) )\\
                 &= A \cap( (B-C) \cup (C-B) )\\
                 &= (A \cap (B-C)) \cup (A \cap (C-B)\\
                 &= ((A \cap B) - (A \cap C) \cup ((A \cap C) - (A \cap B))\\
                 &= (A \cap B) + (A \cap C)\\
                 &= A \times B + A \times C
\end{split}
\end{equation*}

\vspace{3mm}

La distributiva por derecha es similar. Así, $R$ es un anillo.\\

\textit{b) i)} $R$ es conmutativo, pues $A \times B = A \cap B = B \cap A = B \times A$.\\

\textit{ii)} $A \times X = A \cap X = A$. Así, R tiene unidad.\\

\textit{iii)} $A \times A = A \cap A = A$. Esto implica que $R$ es booleano.







\end{proof}