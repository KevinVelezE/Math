\begin{problem}[30] 
Sea $A = \mathbb{Z} \times \mathbb{Z} \times \mathbb{Z} \times ...$ el producto directo de copias de $\mathbb{Z}$ indexado por los enteros positivos ($A$ es un anillo sobre la suma y multiplicación por componentes) y sea $R$ el anillo de todos los homomorfismos de grupos de A a sí mismo. Sea $\varphi$ el elemento de $R$ defininido por $\varphi(a_{1},a_{2},a_{3},...)$. Sea $\psi$ el elemento de $R$ definido por $\psi (a_{1},a_{2},a_{3},...) = (0, a_{1}, a_{2}, a_{3},...)$.\\

a) Pruebe que $\varphi \psi$ es la identidad de R pero $\psi \varphi$ no lo es.\\
b) Exhiba infinitos inversos por derecha para $\varphi$.\\
c) Encuentre un elemento no nulo $\pi$ en $R$ tal que $\varphi \pi = 0$ pero $\pi \varphi \neq 0 $.\\
d) Pruebe que no existe un elemento $\lambda \in R$ no nulo tal que $\lambda \varphi = 0$.\\

\end{problem}

\begin{proof}
a) Tenemos que $\varphi \psi = \varphi \circ \psi$, luego\\
\begin{equation*}
\begin{split}
    \varphi \psi &= \varphi(\psi((a_{1},a_{2},a_{3},...)))\\
                 &= \varphi ((0,a_{1},a_{2},...))\\
                 &= ((a_{1},a_{2},a_{3},...)
\end{split}
\end{equation*}

\vpsace{3mm}

Por otro lado, $\psi \varphi = \psi(\varphi((a_{1},a_{2},a_{3},...))) = \psi((a_{2},a_{3},a_{4},...)) = (0,a_{2},a_{3},...) \neq $ \textbf{0}.\\

b) Sea $\psi_{i}((a_{1},a_{2},a_{3},...)) = (a_{i},a_{1},a_{2},...)$ con $i = 1,2,...$, entonces $\varphi \psi$ es la identidad. Por ejemplo, para $i = 1$ tenemos que 

\begin{equation*}
    \varphi \psi_{1}((a_{1},a_{2},a_{3},...)) = \varphi (a_1
\end{equation*}

\end{proof}