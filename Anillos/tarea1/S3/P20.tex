\begin{problem}[20]
\end{problem}

\begin{proof}
    $I \cap S$ es un subanillo por el ejercicio $7.1.4$, así que solo vamos a mostrar la propiedad de absorción. Si $s \in S$ y $x \in I \cap S$, entonces $sx,xs \in I$ ya qie $I \subseteq R$ es un ideal, y $sx, xs \in S$, ya que $S$ es cerrado bajo la multiplicación y $x \in S$. Por lo que $sx, xs \in I \cap S$, así que $I \cap S \subseteq$ es un ideal.

    Sea $R=\Q$ y $S=\Z$, claramente $S \subseteq R$ es un subanillo. Ahora, consideremos el ideal $J = 2\Z \subseteq S$. Note que $\Q$ tiene solo los ideales $0$ y $\Q$, y que $S \cap 0 = 0$ y $S \cap \Q = S$; en ningún caso es el ideal $J$. 
\end{proof}