\documentclass{amsart}
% Language and encoding
\usepackage[spanish]{babel}
\usepackage[utf8]{inputenc}

% margins
% \usepackage[lmargin=2.5cm, rmargin=2cm]{geometry}
\usepackage[lmargin=3cm, rmargin=2cm, bmargin=2cm, tmargin=2cm]{geometry}

% Math packages
\usepackage{amsmath}
\usepackage{amssymb}
\usepackage{amsthm}

% graphics and color
\usepackage{graphicx}
\usepackage{xcolor}
\usepackage{float}
\usepackage{subfigure}
\usepackage{wrapfig}

% color boxes 
\usepackage[most,many,breakable]{tcolorbox}

% fancy headers
\usepackage{fancyhdr}

% Tikz
\usepackage{tikz}
\usetikzlibrary{angles, calc, arrows, arrows.meta}
\AtBeginEnvironment{tikzpicture}{\shorthandoff{>}\shorthandoff{<}}{}{}

% cancel
\usepackage[makeroom]{cancel}
\newcommand\cancelc[2][black]{\renewcommand\CancelColor{\color{#1}}\cancel{#2}}

% fancy settings
\setlength{\headheight}{18pt}
\pagestyle{fancy}
\fancyhf{}
\fancyhead[L]{\includegraphics[height=5mm]{./figures/logo}}
\fancyfoot[R]{\thepage}

% date format
\renewcommand{\datename}{\emph{Fecha:}}


% ------------------ %

% Theorem environments

% colors for theorem environments
\definecolor{myqsbg}{HTML}{f2fbfc}
\definecolor{myqsfr}{HTML}{191971}

% problem environment
\tcbuselibrary{theorems,skins,hooks}
\newtcbtheorem{question}{Problema}
{
	enhanced,
	breakable,
	colback = myqsbg,
	frame hidden,
	boxrule = 0sp,
	borderline west = {2pt}{0pt}{myqsfr},
	sharp corners,
	detach title,
	before upper = \tcbtitle\par\smallskip,
	coltitle = myqsfr,
	fonttitle = \bfseries\sffamily,
	description font = \mdseries,
	separator sign none,
	segmentation style={solid, myqsfr},
}
{th}

% Environment for problems
\newtheorem{problem}{Problema}

% simplified theorem environment

\newcommand{\qs}[2]{\begin{question}{#1}{}#2\end{question}}

% solution command
\newcommand{\sol}[1]{\noindent\textbf{\textit{Solución:}} #1 \par \hfill $_\square$ }
\newcommand{\dem}[1]{\noindent\textbf{\textit{Demostración:}} #1 \par \hfill $_\square$ }

% shortcuts for some math symbols
\newcommand{\R}{\mathbb{R}}
\newcommand{\N}{\mathbb{N}}
\newcommand{\Z}{\mathbb{Z}}
\newcommand{\Q}{\mathbb{Q}}
\newcommand{\C}{\mathbb{C}}

\newcommand{\Bb}{\mathcal{B}}
\newcommand{\ps}[1]{\mathcal{P}\left(#1\right)}

\newcommand{\dd}{\mathrm{d}}
\newcommand{\ZnZ}{\Z / n\Z}

% shortcuts for some math operators
\newcommand{\abs}[1]{\left\lvert #1 \right\rvert}
\newcommand{\norm}[1]{\left\lVert #1 \right\rVert}

\newcommand{\ol}[1]{\overline{#1}}
\title{Tarea Anillos}
\author{Esteban Ospino, Kevin Velez}
\date{Octubre 2022}

\begin{document}
\maketitle \thispagestyle{fancy}

\section*{7.1 Definiciones básicas y ejemplos}
% \begin{problem}[13]
\end{problem}


\begin{proof} \,
    \begin{enumerate}
        \item Sea $n = a^k b$ para algunos $a$ y $b$.
        Entonces $$ (ab)^k = a^k b^k = (a^k b) b^{k-1} = n b^{k-1} \equiv 0 \mod{n} $$
        
        \item Supongamos que $ \ol{a} \in \ZnZ $ es nilpotente. Entonces $ \ol{a}^k = \ol{0} $ para algún $ k \in \Z^+ $. Entonces $a^k \equiv 0 \mod{n}$, así $n | a^k$. Ahora, si $p$ es un primo divisor de $n$, entonces $p | a^k$ y por lo tanto $p | a$.
        
        Ahora, cada primo divisor de $n$ es un divisor de  $a$. Sea $ n = {p_1}^{\alpha_1} \cdots {p_k}^{\alpha_k} $ y $ a = {p_1}^{\beta_1} \cdots {p_k}^{\beta_k} m$ donde $1 \leq \alpha_i, \beta_i$ para todo $i$ y para algún entero $m$. Sea $s = \max{\alpha_i}$. Entonces
        $$ a^s = \left( {p_1}^{\beta_1} \cdots {p_k}^{\beta_k} m \right)^s = {p_1}^{\beta_1 s} \cdots {p_k}^{\beta_k s} m^s$$
        donde $\beta_i s \geq \alpha i$, por lo que
        $$ a^s = n {p_1}^{\beta_1 s - \alpha_1} \cdot {p_k}^{\beta_k s - \alpha_k} m$$
        y por tanto $ a \equiv 0 \mod{n} $
        
        $ 72 = 2^3 3 ^2 $, entonces los elementos nilpotentes de $ \Z / 72\Z $ son  $\{ 0, 6, 12, 18, 24, 30, 36, 42, 48, 54, 60, 66 \}$.
        
        \item Supongamos que $ \alpha \in R $ es nilpotente. Si $\alpha \neq 0$, entonces existe $ x \in X $ tal que $ \alpha(x) \neq 0 $. Sea $m$ el menor entero positivo tal que $ \alpha(x)^m = 0 $; note que $ m \geq 1 $. Entonces $ \alpha(x) {\alpha(x)}^{m-1} $, donde $ \alpha(x) $ y $ {\alpha(x)}^{m-1} $ no son cero. Por lo que $F$ contiene divisores de cero, lo cual es una contradicción. Por lo que $R$ no contiene elementos nilpotentes diferentes de cero
    \end{enumerate}
\end{proof}



% \begin{problem}[21] Sea X cualquier subconjunto no vacío y sea $\mathcal{P} (X)$ el conjunto de todos los subconjuntos de $X$. Se define la suma y multiplicación sobre $\mathcal{P} (X)$ por \\
\begin{center}
    $A+B = (A - B) \cup (B - A) $ \hspace{3mm} y \hspace{3mm} $A\times B = A \cap B$
\end{center}
\vspace{3mm}

es decir, la suma es la diferencia simétrica de conjuntos y el producto es la intersección.\\

a) Pruebe que $\mathcal{P}(X)$ es un anillo bajo estas operaciones.\\
b) Pruebe que este anillo es conmutativo, tiene identidad y es booleano.\\


\end{problem}

\begin{proof}
a) $(\mathcal{P}(X), + )$ es un grupo abeliano.\\

\textit{i)} Sean $A,B,C \in \mathcal{P}(X)$, $(A+B)+C = A+(B+C)$ puesto que la diferencia simétrica es asociativa. \\

\textit{ii)} Sea $A \in \mathcal{P}(X)$, 

\begin{equation*}
\begin{split}
    A + \emptyset &= (A - \emptyset) \cup (\emptyset - A)\\
                  &= A \cup \emptyset\\
                  &= A
\end{split}
\end{equation*}

Análogamente se tiene que $\emptyset + A = A$. Por tanto $\emptyset$ es el elemento neutro de +.\\

\textit{iii)} Sea $A \in \mathcal{P}(X)$, 
\begin{equation*}
\begin{split}
    A+A &= (A-A) \cup (A-A)\\
        &= \emptyset \cup \emptyset\\
        &= \emptyset
\end{split}
\end{equation*}

esto es, $A$ es su propio inverso.\\

\textit{iv)} Sean $A,B \in \mathcal{P}(X)$,

\begin{equation*}
\begin{split}
    A+B &= (A-B) \cup (B-A)\\
        &= (B-A) \cup (A-B)\\
        &= B + A
\end{split}
\end{equation*}

\vspace{3mm}

$\times$ es asociativo.\\

\begin{equation*}
\begin{split}
    (A\times B)\times C &= (A \cap B)\times C\\
                      &= (A \cap B)\cap C \\
                      &= A \cap (B \cap C)\\
                      &= A \times (B \cap C)\\
                      &= A \times (B \times C)
\end{split}
\end{equation*}

\vspace{3mm}

Valen las propiedades distributivas para $A,B,C \in \mathcal{P}(X)$

\begin{equation*}
\begin{split}
    A\times (B+C) &= A\times( (B-C) \cup (C-B) )\\
                 &= A \cap( (B-C) \cup (C-B) )\\
                 &= (A \cap (B-C)) \cup (A \cap (C-B)\\
                 &= ((A \cap B) - (A \cap C) \cup ((A \cap C) - (A \cap B))\\
                 &= (A \cap B) + (A \cap C)\\
                 &= A \times B + A \times C
\end{split}
\end{equation*}

\vspace{3mm}

La distributiva por derecha es similar. Así, $R$ es un anillo.\\

\textit{b) i)} $R$ es conmutativo, pues $A \times B = A \cap B = B \cap A = B \times A$.\\

\textit{ii)} $A \times X = A \cap X = A$. Así, R tiene unidad.\\

\textit{iii)} $A \times A = A \cap A = A$. Esto implica que $R$ es booleano.







\end{proof}
% \begin{proof}
    \,
    \begin{enumerate}
        \item \, 
        \begin{align*}
            \alpha \overline{\alpha} &= (a+bi+cj+dk)(a-bi-cj-dk) \\
            &= a^2 - abi - acj - adk + bai - b^2i^2 - bcij - bdik + caj \\
            &\ \qquad \qquad - cbji - c^2j^2 - cdjk + dak - daki - dckj - d^2k^2 \\
            &= a^2+b^2+c^2+d^2 \\ 
            &= N(\alpha)
        \end{align*}
        \item \,
        \begin{align*}
            N(\alpha\beta) &= N((a+bi+cj+dk)(x+yi+zj+wk)) \\
            &= N((ax-by-cz-dw) +(ay+bx+cw-dz)i + \\
            &\ \qquad \qquad (az-bw + cx+dy)j +(aw+bz-cy+dx)k) \\
            &= (ax-by-cz-dw)^2 + (ay+bx+cw-dz)^2 + \\
            &\ \qquad \qquad (az-bw+cx+dy)^2  +(aw+bz-cy+dx)^2\\ 
            &= (a^2 + b^2 + c^2 + d^2)(x^2 + y^2 + z^2 + w^2)\\ 
            &= N(\alpha)N(\beta)
        \end{align*}
        
        \item Supongamos que $\alpha$ es una unidad, entonces $\alpha\beta = 1$ para algún $\beta$ en el anillo  de cuaterniones de Hamilton. Note que por la definición de $N$ como una suma de cuadrados, entonces $N(\alpha) \geq 0$ para todo $\alpha \in I$. Ahora
        $$ 1 = N(1) = N(\alpha\beta) = N(\alpha)N(\beta) $$
        y $N(\alpha)$ y $N(\beta)$ son ambos enteros, por lo tanto $N(\alpha) = 1$. Ahora, supongamos que $N(\alpha) = 1$. Entonces
        $$ \alpha \ol{\alpha} = N(\alpha) = 1$$
        Y claramente $\ol{\alpha} \in I$, así que $\alpha$ es una unidad en $I$.

        Supongamos que $\alpha \in I$ es una unidad, entonces
        $$ N(\alpha) = a^2 + b^2 + c^2 + d^3 = 1$$
        si alguno de $a, b, c, d$ es mayor que 2 en valor absoluto, tenemos una contradicción, así que cada uno de $a, b, c, d$ es menor que 2 en valor absoluto. Por lo tanto son $1, 0$ o  $-1$, Si más de uno es mayor que 1 en valor absoluto, tenemos otra contradicción; por lo tanto, a lo mucho uno de ellos puede set $1$, si todos son cero, entonces $\alpha = 0$, por lo que no es una unidad, así que exactamente uno de ellos es 1, entonces $N(\alpha) = 1$, así que $\alpha$ es una unidad. Por lo tanto $\abs{I^*} = 8$, ya que $I^*$ es no abeliano y tiene seis elementos de orden 4, $I^* \cong Q_8$.
    \end{enumerate}
\end{proof}
% \begin{problem}[30] 
Sea $A = \mathbb{Z} \times \mathbb{Z} \times \mathbb{Z} \times ...$ el producto directo de copias de $\mathbb{Z}$ indexado por los enteros positivos ($A$ es un anillo sobre la suma y multiplicación por componentes) y sea $R$ el anillo de todos los homomorfismos de grupos de A a sí mismo. Sea $\varphi$ el elemento de $R$ defininido por $\varphi(a_{1},a_{2},a_{3},...)$. Sea $\psi$ el elemento de $R$ definido por $\psi (a_{1},a_{2},a_{3},...) = (0, a_{1}, a_{2}, a_{3},...)$.\\

a) Pruebe que $\varphi \psi$ es la identidad de R pero $\psi \varphi$ no lo es.\\
b) Exhiba infinitos inversos por derecha para $\varphi$.\\
c) Encuentre un elemento no nulo $\pi$ en $R$ tal que $\varphi \pi = 0$ pero $\pi \varphi \neq 0 $.\\
d) Pruebe que no existe un elemento $\lambda \in R$ no nulo tal que $\lambda \varphi = 0$.\\

\end{problem}

\begin{proof}
a) Tenemos que $\varphi \psi = \varphi \circ \psi$, luego\\
\begin{equation*}
\begin{split}
    \varphi \psi &= \varphi(\psi((a_{1},a_{2},a_{3},...)))\\
                 &= \varphi ((0,a_{1},a_{2},...))\\
                 &= ((a_{1},a_{2},a_{3},...)
\end{split}
\end{equation*}

\vpsace{3mm}

Por otro lado, $\psi \varphi = \psi(\varphi((a_{1},a_{2},a_{3},...))) = \psi((a_{2},a_{3},a_{4},...)) = (0,a_{2},a_{3},...) \neq $ \textbf{0}.\\

b) Sea $\psi_{i}((a_{1},a_{2},a_{3},...)) = (a_{i},a_{1},a_{2},...)$ con $i = 1,2,...$, entonces $\varphi \psi$ es la identidad. Por ejemplo, para $i = 1$ tenemos que 

\begin{equation*}
    \varphi \psi_{1}((a_{1},a_{2},a_{3},...)) = \varphi (a_1
\end{equation*}

\end{proof}

\section*{7.2 Ejemplos: Anillo de polinomios y anillo de matrices}
% \begin{problem}[2]
\end{problem}

\begin{proof}
    Si $b p(x) = 9$ para algún $0\neq b \in R$, entonces es claro que $p(x)$ es un divisor de cero.

    Ahora, supongamos que $p(x)$ es un divisor de cero, esto eso, para algún $q(x) = \sum_{i=0}^{m} b_i x^i$, tenemos que $p(x)q(x)=0$. Podemos elegir el $q(x)$  que tenga grado mínimo que cumpla esta propiedad.

    Vamos a mostrar ahora, por inducción que $a_iq(x) = 0$ para todo $0 \leq i \leq n$

    Para el caso base, note que
    $$ p(x)q(x) = \sum_{k=0}^{n+m} \left(\sum_{i+j = k} a_ib_j\right) x^k = 0 $$
    
    El coeficiente de $x^{n+m}$ en este producto es $a_nb_m$ por un lado, y $0$ por el otro, por lo que $a_nb_m = 0$.

    Ahora, $a_nq(x)p(x) = 0$, y el coeficiente de $x^m$ en $q$ es $a_nb_m=0$. Por lo que el grado de $a_nq(x)$ es estrictamente menor que el grado de $q(x)$; ya que el grado de $q(x)$ es el de grado mínimo entre los polinomio no cero que multiplicados con $p(x)$ dan cero. es un hecho que $a_nq(x) = 0$. Más específicamente, $a_nb_i = 0$ para todo $0 \leq i \leq m$.

    Para el paso inductivo, supongamos que para algún $0 \leq t \leq n$, tenemos que $a_rq(x) == 0$ para todo $t \leq r \leq n$. Ahora
    $$ p(x)q(x) = \sum_{k=0}^{n+m} \left( \sum_{i+j=k} a_ib_j\right) x^k = 0 $$
    
    Por un lado, el coeficiente de $x^{m+t}$ es $\sum_{i+j=m+t}a_ib_j$, y por el otro lado, es $0$, por lo que 
    $$ \sum_{i+j=m+t} a_ib_j = 0$$

    Por la hipótesis de inducción, si $i \geq t$, entonces $a_ib_j = 0$, por lo que todos los términos tal que $i \geq t$ son cero. Si $i<t$, entonces tenemos que $j>m$, una contradicción. Por lo que $a_ib_m = 0$. Como en el caso base, 
    $$ a_tq(x)p(x) = 0$$
    y $a_tq(x)$ tiene grado estrictamente menor que el grado de $q(x)$, Así que, por la minimalidad, $a_tq(t)=0$.

    Por inducción, $a_iq(x) = 0$ para todo $0 \leq i \leq n$. En particular, $a_ib_m =0$. Por lo que $b_mp(x) = 0$
\end{proof}
% Hola
% \begin{problem}[7]
\end{problem}

\begin{proof}
    Primero, probemos el siguiente lema\\
    \textit{\textbf{Lema:}} $E_{p,q}E_{s,t} = E_{p,t}$ si $q=s$ y cero en otros casos.\\
    \textit{\textbf{Demostración:}} si $q=s$, entonces la $p$-esima fila de $E_{p,q}E_{s,t}$ es la $s$-esima fila de $E_{s,t}$ y todas las otras entradas son 0.

    Por lo que $E_{p,q}E_{s,t} = E_{p,t}$. Si $q\neq s$, entonces la $p$-esima fila de $E_{p,q}E_{s,t}$ es la $q$-esima fila de $E_{s,t}$ las cuales son todas cero, y todas las otras entradas cero, por lo que $E_{p,q}E_{s,t} = 0$.

    Ahora, supongamos que $B = [b_{i,j}] \in Z(M_n(R))$. note que la entrada $(p,t)$ de $E_{p,q}BE_{s,t} = E_{p,q}E_{s,t}Bisb_{q,s}$
    
    Por el lema, si $q \neq s$, entonces $b_{q,s} = 0$. Por lo que $B$ es una matriz diagonal. Ahora, so $q=s$, entonces la entrada $(p,t)$ de $E_{p,t}B$ es $b_{q,q}$ por un lado y $b_{t,t}$ en el otro, ya que la $p$-esima fila de $E_{p,t}B$ es la $t$-esima fila de $B$. Por lo que $b_{q,q}=b_{t,t}$ para toda elección de $q$ y $t$. por lo que $B=bI$ para algún $b\in R$, y tenemos que $Z(M_n(R)) \subseteq \{ rI \ |\ r \in R \}$

    Por el contrario, $(rI)A = rA = Ar = A(rI)$. por lo que $Z(M_n(R)) = \{ rI \ |\ r \in R \}$
\end{proof}

\section*{7.3 Homomorfismo de anillos y anillo cociente}
% \input{S3/P4}
% \begin{problem}[14]
\end{problem}
\newcommand{\HH}{\mathbb{H}}
\begin{proof}
    Definimos $\varphi: \HH \to M_4(\R)$ de la siguiente manera
    $$
    a+bi+cj+dk \mapsto
    \begin{bmatrix}
        a & b & c & d \\
        -b & a & -d & c \\
        -c & d & a & -b \\
        -d & -c & b & a
    \end{bmatrix}
    $$
    
    Vamos a mostrar que esta función es un homomorfismo inyectivo de anillos. Con ese fin, sea $\alpha = a_1 + b_1 i + c_1 j + d_1 k$ y $\beta = a_2 + b_2 i + c_2 j + d_2 k$. Entonces tenemos que
    \begin{align*}
        \varphi(\alpha + \beta) &= \varphi((a_1 + b_1i + c_1j + d_1k)+(a_2 + b_2i + c_2j + d_2k)) \\
        &= \varphi((a_1 + a_2) + (b_1+b_2)i + (c_1+c_2)j + (d_1+d_2)k)\\  
        &=
        \begin{bmatrix}
            a_1+a_2 & b_1+b_2 & c_1+c_2 & d_1+d_2 \\
            -b_1-b_2 & a_1+a_2 & -d_1-d_2 & c_1+c_2 \\
            -c_1-c_2 & d_1+d_2 & a_1+a_2 & -b_1-b_2 \\
            -d_1-d_2 & -c_1-c_2 & b_1+b_2 & a_1+a_2 
        \end{bmatrix} \\
        &=
        \begin{bmatrix}
            a_1 & b_1 & c_1 & d_1 \\
            -b_1 & a_1 & -d_1 & c_1 \\
            -c_1 & d_1 & a_1 & -b_1 \\
            -d_1 & -c_1 & b_1 & a_1
        \end{bmatrix}+
        \begin{bmatrix}
            a_2 & b_2 & c_2 & d_2 \\
            -b_2 & a_2 & -d_2 & c_2 \\
            -c_2 & d_2 & a_2 & -b_2 \\
            -d_2 & -c_2 & b_2 & a_2
        \end{bmatrix} \\
        &= \varphi(a_1 + b_1i + c_1j + d_1k) + \varphi(a_2 + b_2i + c_2j + d_2k)\\
        &= \varphi(\alpha) + \varphi(\beta)
    \end{align*}
    Y de manera similar se muestra que se cumple que $\varphi(\alpha \beta) = \varphi(\alpha) \varphi(\beta)$.

    Por lo que, $\varphi$ es un homomorfismo de anillo.

    Supongamos ahora que $\alpha = a + bi +cj +dk \in \ker \varphi$; entonces tenemos que $a = b = c = d = 0$, por lo que $\alpha = 0$, esto es, $\varphi$ es inyectiva, y por tanto
    $$ \HH \cong \mbox{Im}(\varphi) $$
    donde $\mbox{Im}(\varphi)$ es subanillo de $M_4(\R)$.

\end{proof}
% \input{S3/P15}
% \begin{problem}[20]
\end{problem}

\begin{proof}
    $I \cap S$ es un subanillo por el ejercicio $7.1.4$, así que solo vamos a mostrar la propiedad de absorción. Si $s \in S$ y $x \in I \cap S$, entonces $sx,xs \in I$ ya qie $I \subseteq R$ es un ideal, y $sx, xs \in S$, ya que $S$ es cerrado bajo la multiplicación y $x \in S$. Por lo que $sx, xs \in I \cap S$, así que $I \cap S \subseteq$ es un ideal.

    Sea $R=\Q$ y $S=\Z$, claramente $S \subseteq R$ es un subanillo. Ahora, consideremos el ideal $J = 2\Z \subseteq S$. Note que $\Q$ tiene solo los ideales $0$ y $\Q$, y que $S \cap 0 = 0$ y $S \cap \Q = S$; en ningún caso es el ideal $J$. 
\end{proof}
% \input{S3/P26}
% \begin{problem}[30] 
Sea $A = \mathbb{Z} \times \mathbb{Z} \times \mathbb{Z} \times ...$ el producto directo de copias de $\mathbb{Z}$ indexado por los enteros positivos ($A$ es un anillo sobre la suma y multiplicación por componentes) y sea $R$ el anillo de todos los homomorfismos de grupos de A a sí mismo. Sea $\varphi$ el elemento de $R$ defininido por $\varphi(a_{1},a_{2},a_{3},...)$. Sea $\psi$ el elemento de $R$ definido por $\psi (a_{1},a_{2},a_{3},...) = (0, a_{1}, a_{2}, a_{3},...)$.\\

a) Pruebe que $\varphi \psi$ es la identidad de R pero $\psi \varphi$ no lo es.\\
b) Exhiba infinitos inversos por derecha para $\varphi$.\\
c) Encuentre un elemento no nulo $\pi$ en $R$ tal que $\varphi \pi = 0$ pero $\pi \varphi \neq 0 $.\\
d) Pruebe que no existe un elemento $\lambda \in R$ no nulo tal que $\lambda \varphi = 0$.\\

\end{problem}

\begin{proof}
a) Tenemos que $\varphi \psi = \varphi \circ \psi$, luego\\
\begin{equation*}
\begin{split}
    \varphi \psi &= \varphi(\psi((a_{1},a_{2},a_{3},...)))\\
                 &= \varphi ((0,a_{1},a_{2},...))\\
                 &= ((a_{1},a_{2},a_{3},...)
\end{split}
\end{equation*}

\vpsace{3mm}

Por otro lado, $\psi \varphi = \psi(\varphi((a_{1},a_{2},a_{3},...))) = \psi((a_{2},a_{3},a_{4},...)) = (0,a_{2},a_{3},...) \neq $ \textbf{0}.\\

b) Sea $\psi_{i}((a_{1},a_{2},a_{3},...)) = (a_{i},a_{1},a_{2},...)$ con $i = 1,2,...$, entonces $\varphi \psi$ es la identidad. Por ejemplo, para $i = 1$ tenemos que 

\begin{equation*}
    \varphi \psi_{1}((a_{1},a_{2},a_{3},...)) = \varphi (a_1
\end{equation*}

\end{proof}
% \begin{problem}[33]
\end{problem}

\begin{proof}
    \begin{enumerate}
        \item[a)] Sea $M \subseteq R$ un ideal maximal, y supongamos $M \neq M_c$ para todo $c \in [0,1]$. Entonces para todo $c$, existe una función $f_c \in M$ tal que $f_c(c) \neq 0$. Sin perdida de generalidad, asumamos que $f_c(c) > 0$. Como $f_c$ es continua, existe un número real positivo $\varepsilon_C > 0$ tal que $f_c[(c-\varepsilon_c, c+\varepsilon_c)] = 0$. Claramente, el conjunto 
        $$ \left\{ (c-\varepsilon_c, c+\varepsilon_c) \cap [0,1] \, | \, c \in [0,1] \right\}$$
        cubre $[0,1]$, ya que $[0,1]$ es compacto, esta cubierta tiene una subcubierta finita;  $K \subseteq [0,1]$ es finito y 
        $$ \left\{ (c-\varepsilon_c, c+\varepsilon_c) \cap [0,1] \, | \, c \in K \right\} $$
        cubre $[0,1]$. Ahora, para cada $c \in K$, definimos $u_c(x) = 1 + (x-c)/\varepsilon_c$ si $x \in (c-\varepsilon_c,c) \cap [0,1]$, $1 + (x-c)/\varepsilon_c$ si $x \in [c, c+\varepsilon_c] \cap [0,1]$, y cero en otros caso.

        Evidentemente $u_c \in R$ y $u_c$ desaparece fuera de $(c-\varepsilon_c, c+\varepsilon_x)$. Consideremos $g = \sum u_c f_c$, como $f_c \in M$, tenemos que $g \in M$. Sin embargo, para todo $x \in [0,1]$, $g(x)$ es positiva ya que cada $u_c(x)f_c(x)$ es no negativo y algunos $u_c(x)f_c(x)$ son positivos. Por lo que $g(x) > 0$ para todo $x$ y por tanto $1/g \in R$. Pero entonces $M$ contiene una unidad, lo cual es una contradicción, por lo que $M = M_c$ para algún $c \in [0,1]$.

        \item[b)] Supongamos que $b \neq c$, notemos que $x-b \in M_b$ pero $(x-b)(c) = c-b \neq 0$ así que $x-b \notin M_c$, por lo que $M_b \neq M_c$.
        
        \item[c)] Supongamos que $M_c = (x-c)$. Entonces, en particular, $\abs{c-x} = f(x)(x-c)$ para algún $f(x) \in R$. Para $x \neq c$ tenemos que $f(c) = \frac{\abs{x-c}}{x-c}$. Notemos que como $x$ se acerca a $c$ por la derecha, $f(x)$ se acerca a $-\infty$, mientras que $f(x)$ se acerca a $+\infty$ cuando $x$ se acerca a $c$ por la izquierda. En particular, ninguna extensión de $f$ está en $R$. Por lo que $M_c$ no es generado por $x-c$.
    
        \item[d)] supongamos que $M_c = (A)$ es finitamente generado, con $A = \left\{ a_i(x) | 1 \leq i \leq n \right\}$. Sea $f = \sum \abs{a_i}$; entonces $\sqrt{f}$ es continua en $[0,1]$. Más aún, tenemos que $\sqrt{f} \in M_c$. por lo que $\sqrt{f} = \sum r_ia_i$ para alguna función continua $r_i \in R$. Si dejamos $r = \sum \abs{r_i}$, tenemos que
        $$ \sqrt{f}(x) = \sum r_i(x)a_i(x) \leq \sum |r_i(x)||a_i(x)| \leq r(x)f(x) $$
        Notemos que para cada $b \neq c$, debe existir una función $a_i$ tal que $a_i(b) \neq 0$, Como en otros casos tenemos que $h(b) = 0$ para todo $h \in M_c$, y en particular $x-c \in M_c$. Por lo que $c$ es el único cero de $f$. como $\sqrt{f(x)} \leq r(x)f(x)$ para $x \neq c$, tenemos que $r(x) \geq 1 / \sqrt{f(x)}$. Si $x$ se acerca a $c$, $f(x)$ se aproxima a $0$, por lo que $1 / \sqrt{f(x)}$ no es acotado. Pero entonces $r(c)$ no existe, una contradicción, ya que $r(x) \in R$. Por lo que $M_c$ no es finitamente generado.
    \end{enumerate}
\end{proof}
% \begin{problem}[34]
\end{problem}

\begin{proof} \, 
    \begin{enumerate}
        \item[a)] Primero, veamos que $I+J$ es un ideal de $R$. Sea $a_1 + b_1, a_2 + b_2 \in I + J$ para algunos $a_i \in I$ y $b_i \in J$. Entonces
        $$ (a_1+b_1)-(a_2+b_2) = (a_1-a_2) + (b_1-b_2) \in I +J $$
        ya que $I$ y $J$ son cerrados bajo la sustracción. Ahora, sea $r \in R$. Tenemos que 
        $$ r(a+b) = ra + rb \in I + J$$
        ya que $I$ y $J$ absorben de $R$ por la derecha, y de manera similar $(a+b)r \in I+J$. Por lo tanto, $I+J$ es un ideal de $R$.

        Ahora, vamos a mostrar que $I+J$ es el más pequeño que contiene a $I$ y a $J$. Supongamos que $K$ es un ideal de $R$ con $I,J \subseteq K$. Si $a+b \in I+J$, entonces como $K$ es cerrado bajo la adición, tenemos que $a+b \in K$, por lo que $I+J \subseteq K$.

        \item[b)] Primero, sea $\alpha, \beta \in IJ$, con $\alpha = \sum a_i b_i$ y $\beta = \sum c_i d_i$. donde $a_i, c_i \in I$ y $b_i, d_i \in J$. Claramente $\alpha\beta \in IJ$. Ahora, sea $r \in R$. como $I$ es un ideal de R,
        $$ r\alpha = \sum (r a_i) b_i \in IJ$$
        Similarmente, $\alpha r \in IJ$. Por lo tanto, $IJ$ es un ideal de $R$.

        Ahora, consideremos de nuevo $\alpha = \sum a_i b_i$, como $a_i \in I$ e $I$ es un ideal, $a_i b_i \in I$, y por tanto, $\alpha \in I$, de mismo modo, $\alpha \in J$. Por lo tanto, $IJ \subseteq I \cap J$.

        \item[c)] Sea $R = \Z$, $I = 2\Z$ y $J = 4\Z$. Evidentemente
        $$ (2\Z)(4\Z) = 8\Z$$
        mientra que
        $$ 2\Z \cap 4\Z = 4\Z$$
        Por lo que no es generalmente cierto que $IJ = I \cap J$.

        \item[d)] Supongamos que $I+J = R$ y que $R$ es conmutativo, sabemos que $IJ \subseteq I \cap J$.
        
        Notemos que 
        $$ (I \cap J)(I+J) = (I \cap J)R = I \cap J $$
        Por lo que, si $z \in I \cap J$, entonces $z = w(x+y)$ para algún $x \in I$ y $y \in J$. y $w \in I \cap J$. Entonces
        $$ z = xw + wy \in IJ $$
        Por lo tanto, $IJ = I \cap J$.
    \end{enumerate}
\end{proof}

\section*{7.4 Propiedades de ideales}
% \input{S4/P10}
% \begin{problem}[14]
\end{problem}
\newcommand{\HH}{\mathbb{H}}
\begin{proof}
    Definimos $\varphi: \HH \to M_4(\R)$ de la siguiente manera
    $$
    a+bi+cj+dk \mapsto
    \begin{bmatrix}
        a & b & c & d \\
        -b & a & -d & c \\
        -c & d & a & -b \\
        -d & -c & b & a
    \end{bmatrix}
    $$
    
    Vamos a mostrar que esta función es un homomorfismo inyectivo de anillos. Con ese fin, sea $\alpha = a_1 + b_1 i + c_1 j + d_1 k$ y $\beta = a_2 + b_2 i + c_2 j + d_2 k$. Entonces tenemos que
    \begin{align*}
        \varphi(\alpha + \beta) &= \varphi((a_1 + b_1i + c_1j + d_1k)+(a_2 + b_2i + c_2j + d_2k)) \\
        &= \varphi((a_1 + a_2) + (b_1+b_2)i + (c_1+c_2)j + (d_1+d_2)k)\\  
        &=
        \begin{bmatrix}
            a_1+a_2 & b_1+b_2 & c_1+c_2 & d_1+d_2 \\
            -b_1-b_2 & a_1+a_2 & -d_1-d_2 & c_1+c_2 \\
            -c_1-c_2 & d_1+d_2 & a_1+a_2 & -b_1-b_2 \\
            -d_1-d_2 & -c_1-c_2 & b_1+b_2 & a_1+a_2 
        \end{bmatrix} \\
        &=
        \begin{bmatrix}
            a_1 & b_1 & c_1 & d_1 \\
            -b_1 & a_1 & -d_1 & c_1 \\
            -c_1 & d_1 & a_1 & -b_1 \\
            -d_1 & -c_1 & b_1 & a_1
        \end{bmatrix}+
        \begin{bmatrix}
            a_2 & b_2 & c_2 & d_2 \\
            -b_2 & a_2 & -d_2 & c_2 \\
            -c_2 & d_2 & a_2 & -b_2 \\
            -d_2 & -c_2 & b_2 & a_2
        \end{bmatrix} \\
        &= \varphi(a_1 + b_1i + c_1j + d_1k) + \varphi(a_2 + b_2i + c_2j + d_2k)\\
        &= \varphi(\alpha) + \varphi(\beta)
    \end{align*}
    Y de manera similar se muestra que se cumple que $\varphi(\alpha \beta) = \varphi(\alpha) \varphi(\beta)$.

    Por lo que, $\varphi$ es un homomorfismo de anillo.

    Supongamos ahora que $\alpha = a + bi +cj +dk \in \ker \varphi$; entonces tenemos que $a = b = c = d = 0$, por lo que $\alpha = 0$, esto es, $\varphi$ es inyectiva, y por tanto
    $$ \HH \cong \mbox{Im}(\varphi) $$
    donde $\mbox{Im}(\varphi)$ es subanillo de $M_4(\R)$.

\end{proof}
% \input{S4/P22}
% \begin{proof}
    \,
    \begin{enumerate}
        \item \, 
        \begin{align*}
            \alpha \overline{\alpha} &= (a+bi+cj+dk)(a-bi-cj-dk) \\
            &= a^2 - abi - acj - adk + bai - b^2i^2 - bcij - bdik + caj \\
            &\ \qquad \qquad - cbji - c^2j^2 - cdjk + dak - daki - dckj - d^2k^2 \\
            &= a^2+b^2+c^2+d^2 \\ 
            &= N(\alpha)
        \end{align*}
        \item \,
        \begin{align*}
            N(\alpha\beta) &= N((a+bi+cj+dk)(x+yi+zj+wk)) \\
            &= N((ax-by-cz-dw) +(ay+bx+cw-dz)i + \\
            &\ \qquad \qquad (az-bw + cx+dy)j +(aw+bz-cy+dx)k) \\
            &= (ax-by-cz-dw)^2 + (ay+bx+cw-dz)^2 + \\
            &\ \qquad \qquad (az-bw+cx+dy)^2  +(aw+bz-cy+dx)^2\\ 
            &= (a^2 + b^2 + c^2 + d^2)(x^2 + y^2 + z^2 + w^2)\\ 
            &= N(\alpha)N(\beta)
        \end{align*}
        
        \item Supongamos que $\alpha$ es una unidad, entonces $\alpha\beta = 1$ para algún $\beta$ en el anillo  de cuaterniones de Hamilton. Note que por la definición de $N$ como una suma de cuadrados, entonces $N(\alpha) \geq 0$ para todo $\alpha \in I$. Ahora
        $$ 1 = N(1) = N(\alpha\beta) = N(\alpha)N(\beta) $$
        y $N(\alpha)$ y $N(\beta)$ son ambos enteros, por lo tanto $N(\alpha) = 1$. Ahora, supongamos que $N(\alpha) = 1$. Entonces
        $$ \alpha \ol{\alpha} = N(\alpha) = 1$$
        Y claramente $\ol{\alpha} \in I$, así que $\alpha$ es una unidad en $I$.

        Supongamos que $\alpha \in I$ es una unidad, entonces
        $$ N(\alpha) = a^2 + b^2 + c^2 + d^3 = 1$$
        si alguno de $a, b, c, d$ es mayor que 2 en valor absoluto, tenemos una contradicción, así que cada uno de $a, b, c, d$ es menor que 2 en valor absoluto. Por lo tanto son $1, 0$ o  $-1$, Si más de uno es mayor que 1 en valor absoluto, tenemos otra contradicción; por lo tanto, a lo mucho uno de ellos puede set $1$, si todos son cero, entonces $\alpha = 0$, por lo que no es una unidad, así que exactamente uno de ellos es 1, entonces $N(\alpha) = 1$, así que $\alpha$ es una unidad. Por lo tanto $\abs{I^*} = 8$, ya que $I^*$ es no abeliano y tiene seis elementos de orden 4, $I^* \cong Q_8$.
    \end{enumerate}
\end{proof}
% \begin{problem}[30] 
Sea $A = \mathbb{Z} \times \mathbb{Z} \times \mathbb{Z} \times ...$ el producto directo de copias de $\mathbb{Z}$ indexado por los enteros positivos ($A$ es un anillo sobre la suma y multiplicación por componentes) y sea $R$ el anillo de todos los homomorfismos de grupos de A a sí mismo. Sea $\varphi$ el elemento de $R$ defininido por $\varphi(a_{1},a_{2},a_{3},...)$. Sea $\psi$ el elemento de $R$ definido por $\psi (a_{1},a_{2},a_{3},...) = (0, a_{1}, a_{2}, a_{3},...)$.\\

a) Pruebe que $\varphi \psi$ es la identidad de R pero $\psi \varphi$ no lo es.\\
b) Exhiba infinitos inversos por derecha para $\varphi$.\\
c) Encuentre un elemento no nulo $\pi$ en $R$ tal que $\varphi \pi = 0$ pero $\pi \varphi \neq 0 $.\\
d) Pruebe que no existe un elemento $\lambda \in R$ no nulo tal que $\lambda \varphi = 0$.\\

\end{problem}

\begin{proof}
a) Tenemos que $\varphi \psi = \varphi \circ \psi$, luego\\
\begin{equation*}
\begin{split}
    \varphi \psi &= \varphi(\psi((a_{1},a_{2},a_{3},...)))\\
                 &= \varphi ((0,a_{1},a_{2},...))\\
                 &= ((a_{1},a_{2},a_{3},...)
\end{split}
\end{equation*}

\vpsace{3mm}

Por otro lado, $\psi \varphi = \psi(\varphi((a_{1},a_{2},a_{3},...))) = \psi((a_{2},a_{3},a_{4},...)) = (0,a_{2},a_{3},...) \neq $ \textbf{0}.\\

b) Sea $\psi_{i}((a_{1},a_{2},a_{3},...)) = (a_{i},a_{1},a_{2},...)$ con $i = 1,2,...$, entonces $\varphi \psi$ es la identidad. Por ejemplo, para $i = 1$ tenemos que 

\begin{equation*}
    \varphi \psi_{1}((a_{1},a_{2},a_{3},...)) = \varphi (a_1
\end{equation*}

\end{proof}
% \begin{problem}[33]
\end{problem}

\begin{proof}
    \begin{enumerate}
        \item[a)] Sea $M \subseteq R$ un ideal maximal, y supongamos $M \neq M_c$ para todo $c \in [0,1]$. Entonces para todo $c$, existe una función $f_c \in M$ tal que $f_c(c) \neq 0$. Sin perdida de generalidad, asumamos que $f_c(c) > 0$. Como $f_c$ es continua, existe un número real positivo $\varepsilon_C > 0$ tal que $f_c[(c-\varepsilon_c, c+\varepsilon_c)] = 0$. Claramente, el conjunto 
        $$ \left\{ (c-\varepsilon_c, c+\varepsilon_c) \cap [0,1] \, | \, c \in [0,1] \right\}$$
        cubre $[0,1]$, ya que $[0,1]$ es compacto, esta cubierta tiene una subcubierta finita;  $K \subseteq [0,1]$ es finito y 
        $$ \left\{ (c-\varepsilon_c, c+\varepsilon_c) \cap [0,1] \, | \, c \in K \right\} $$
        cubre $[0,1]$. Ahora, para cada $c \in K$, definimos $u_c(x) = 1 + (x-c)/\varepsilon_c$ si $x \in (c-\varepsilon_c,c) \cap [0,1]$, $1 + (x-c)/\varepsilon_c$ si $x \in [c, c+\varepsilon_c] \cap [0,1]$, y cero en otros caso.

        Evidentemente $u_c \in R$ y $u_c$ desaparece fuera de $(c-\varepsilon_c, c+\varepsilon_x)$. Consideremos $g = \sum u_c f_c$, como $f_c \in M$, tenemos que $g \in M$. Sin embargo, para todo $x \in [0,1]$, $g(x)$ es positiva ya que cada $u_c(x)f_c(x)$ es no negativo y algunos $u_c(x)f_c(x)$ son positivos. Por lo que $g(x) > 0$ para todo $x$ y por tanto $1/g \in R$. Pero entonces $M$ contiene una unidad, lo cual es una contradicción, por lo que $M = M_c$ para algún $c \in [0,1]$.

        \item[b)] Supongamos que $b \neq c$, notemos que $x-b \in M_b$ pero $(x-b)(c) = c-b \neq 0$ así que $x-b \notin M_c$, por lo que $M_b \neq M_c$.
        
        \item[c)] Supongamos que $M_c = (x-c)$. Entonces, en particular, $\abs{c-x} = f(x)(x-c)$ para algún $f(x) \in R$. Para $x \neq c$ tenemos que $f(c) = \frac{\abs{x-c}}{x-c}$. Notemos que como $x$ se acerca a $c$ por la derecha, $f(x)$ se acerca a $-\infty$, mientras que $f(x)$ se acerca a $+\infty$ cuando $x$ se acerca a $c$ por la izquierda. En particular, ninguna extensión de $f$ está en $R$. Por lo que $M_c$ no es generado por $x-c$.
    
        \item[d)] supongamos que $M_c = (A)$ es finitamente generado, con $A = \left\{ a_i(x) | 1 \leq i \leq n \right\}$. Sea $f = \sum \abs{a_i}$; entonces $\sqrt{f}$ es continua en $[0,1]$. Más aún, tenemos que $\sqrt{f} \in M_c$. por lo que $\sqrt{f} = \sum r_ia_i$ para alguna función continua $r_i \in R$. Si dejamos $r = \sum \abs{r_i}$, tenemos que
        $$ \sqrt{f}(x) = \sum r_i(x)a_i(x) \leq \sum |r_i(x)||a_i(x)| \leq r(x)f(x) $$
        Notemos que para cada $b \neq c$, debe existir una función $a_i$ tal que $a_i(b) \neq 0$, Como en otros casos tenemos que $h(b) = 0$ para todo $h \in M_c$, y en particular $x-c \in M_c$. Por lo que $c$ es el único cero de $f$. como $\sqrt{f(x)} \leq r(x)f(x)$ para $x \neq c$, tenemos que $r(x) \geq 1 / \sqrt{f(x)}$. Si $x$ se acerca a $c$, $f(x)$ se aproxima a $0$, por lo que $1 / \sqrt{f(x)}$ no es acotado. Pero entonces $r(c)$ no existe, una contradicción, ya que $r(x) \in R$. Por lo que $M_c$ no es finitamente generado.
    \end{enumerate}
\end{proof}

\section*{7.5 Anillos de fracciones}
% \input{S5/P3}
Hola

\section*{7.6 Teorema chino de los residuos}
% \begin{problem}[2]
\end{problem}

\begin{proof}
    Si $b p(x) = 9$ para algún $0\neq b \in R$, entonces es claro que $p(x)$ es un divisor de cero.

    Ahora, supongamos que $p(x)$ es un divisor de cero, esto eso, para algún $q(x) = \sum_{i=0}^{m} b_i x^i$, tenemos que $p(x)q(x)=0$. Podemos elegir el $q(x)$  que tenga grado mínimo que cumpla esta propiedad.

    Vamos a mostrar ahora, por inducción que $a_iq(x) = 0$ para todo $0 \leq i \leq n$

    Para el caso base, note que
    $$ p(x)q(x) = \sum_{k=0}^{n+m} \left(\sum_{i+j = k} a_ib_j\right) x^k = 0 $$
    
    El coeficiente de $x^{n+m}$ en este producto es $a_nb_m$ por un lado, y $0$ por el otro, por lo que $a_nb_m = 0$.

    Ahora, $a_nq(x)p(x) = 0$, y el coeficiente de $x^m$ en $q$ es $a_nb_m=0$. Por lo que el grado de $a_nq(x)$ es estrictamente menor que el grado de $q(x)$; ya que el grado de $q(x)$ es el de grado mínimo entre los polinomio no cero que multiplicados con $p(x)$ dan cero. es un hecho que $a_nq(x) = 0$. Más específicamente, $a_nb_i = 0$ para todo $0 \leq i \leq m$.

    Para el paso inductivo, supongamos que para algún $0 \leq t \leq n$, tenemos que $a_rq(x) == 0$ para todo $t \leq r \leq n$. Ahora
    $$ p(x)q(x) = \sum_{k=0}^{n+m} \left( \sum_{i+j=k} a_ib_j\right) x^k = 0 $$
    
    Por un lado, el coeficiente de $x^{m+t}$ es $\sum_{i+j=m+t}a_ib_j$, y por el otro lado, es $0$, por lo que 
    $$ \sum_{i+j=m+t} a_ib_j = 0$$

    Por la hipótesis de inducción, si $i \geq t$, entonces $a_ib_j = 0$, por lo que todos los términos tal que $i \geq t$ son cero. Si $i<t$, entonces tenemos que $j>m$, una contradicción. Por lo que $a_ib_m = 0$. Como en el caso base, 
    $$ a_tq(x)p(x) = 0$$
    y $a_tq(x)$ tiene grado estrictamente menor que el grado de $q(x)$, Así que, por la minimalidad, $a_tq(t)=0$.

    Por inducción, $a_iq(x) = 0$ para todo $0 \leq i \leq n$. En particular, $a_ib_m =0$. Por lo que $b_mp(x) = 0$
\end{proof}
% \begin{problem}[6]
\end{problem}

\begin{proof} \,
    \begin{enumerate}
        \item[a)] Si $ a \equiv b \mod{2}$, entonces
        $$ a + bi = \left( \frac{a+b}{2} + \frac{b-a}{2}i \right)(1+i) \in (1+i) $$
        Por lo que $ \ol{a+bi} = \ol{0} $. Si $ a \not\equiv b \mod{2}$, entonces $a+1 \equiv b \mod{2}$, por lo que $ \ol{a+1 + bi} = \ol{a+bi} + \ol{1} = \ol{0} + \ol{1} = \ol{1}  $. Estos son los únicos dos casos posibles, así que $(1+i) = \left\{\ol{0}, \ol{1} \right\}$, como es un dominio con dos elementos, entonces es un cuerpo.

        \item[b)] Primero veamos la cantidad de elementos de $ \Z[i]/(q) $, Sea $a + bi \in \Z[i]$, entonces $\ol{a+bi} = \ol{a} + \ol{b} \cdot \ol{i}$, por lo que, tanto $\ol{a}$ como $\ol{b}$ tienen $q$ posibles valores para tomar, así, la cantidad de elementos de $ \Z[i]/(q) $ es $q\cdot q = q^2$.
        
        Por otro lado, como $q \equiv 3 \mod{4}$, entonces $q$ es irreducible en $\Z[i]$, como $Z[i]$ es dominio de ideales principales, entonces $q$ es primo y por tanto $(q)$ es un ideal maximal, lo cual implica que $\Z[i]/(q)$ es cuerpo.

        \item[c)]
    \end{enumerate}

\end{proof}

\end{document}
