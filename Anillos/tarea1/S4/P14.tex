\begin{problem}[14]
\end{problem}
\newcommand{\gr}{\mbox{gr}}
\begin{proof}
    Primero, probemos el siguiente lema: \\
    \textit{\textbf{Lema:}} Sea $R$ un anillo conmutativo con $1 \neq 0$, Si $f(x) \in R[x]$ es mónico, entonces para todo $g(x) \in R[x]$, se tiene que $\gr(fg) = \gr(f) + \gr(g)$ \\
    \textit{\textbf{Prueba:}} La desigualdad $\gr(fg) \leq \gr(f) + \gr(g)$ siempre se cumple. Escribimos $f(x) = x^n + f'(x)$ y $g(x) = b_mx^m + g'(x)$, donde $f'$ y $g'$ tienen grado menor que $f$ y $g$ respectivamente, Entonces 
    $$ (fg)(x) = b_mx^{n+m} + x^ng^\prime(x) + x^mf^\prime(x) + (g^\prime f^\prime)(x); $$
    ya que $b_m \neq 0$, el grado de $fg$ es la suma de los grados de $f$ y $g$ \hfill $_\blacksquare$

    \begin{enumerate}
        \item[a)] Ahora, sea $\ol{g(x)} \in R[x] / (f(x))$ y escribimos $f(x) = x^n - f'(x)$, donde el grado de $f'$ es menor que $n$. Procedemos por inducción en el grado de $g$.
        
        Para el caso base, si el grado de $g$ es menor que $n$, entonces $g(x)$ es su misma representación $\ol{g(x)}$. Para el paso inductivo, supongamos que para algún $m \geq n-1$, si $h(x)$ es un polinomio de grado a lo sumo $m$, entonces existe $p(x)$ de grado a lo sumo $-1$ tal que $\ol{h(x)} = \ol{p(x)}$. Sea $g(x)$ de grado $m+1$, escribimos $g(x) = b_{m+1}x^{m+1} + g'(x)$, donde el grado de $g'$ es a lo sumo $m$. Notemos que, como $\ol{f(x)} = 0$, entonces $\ol{x^n} = \ol{f'}$. Ahora

        \begin{align*}
            \overline{g(x)} &= \overline{b_{m+1} x^{m+1} + g^\prime(x)}\\
            &= \overline{b_{m+1}x^{m-n+1}}\overline{x^n} + \overline{g^\prime(x)}\\ 
            &= \overline{b_{m+1}x^{m-n+1}}\overline{f^\prime(x)} + \overline{g^\prime(x)} \\
            &= \overline{b_{m+1}x^{m-n+1}f^\prime(x) + g^\prime(x)}
        \end{align*}

        Notemos que el grado de $f'$ es a lo sumo $n$, así que el grado de $x^{m-n+1}f'(x)$ es a lo sumo $m$. Por lo que, por la hipótesis de inducción, tenemos que $\ol{g(x)} = \ol{p(x)} $ para algún polinomio $p(x)$ de grado menor que $n$.
        
        \item[b)] supongamos que $p(x), q(x) \in R[x]$ tienen grado menor que $n$ y que $p(x) \neq q(x)$. Supongamos que $\ol{p(x)} = \ol{q(x)}$. Entonces $p(x) - q(x) = f(x)g(x)$ para algún $g(x) \in R[x]$. Notemos que, sin embargo, $p(x) - q(x)$ tiene grado menor que $n$ mientras que $f(x)g(x)$ tiene grado mayor que $n$. Lo cual es una contradicción.
        
        \item[c)] Supongamos que $f(x) = a(x)b(x)$, donde tanto $a$ como $b$ tienen grado menor que $n$. Entonces
        $$ \ol{a(x)b(x)} = \ol{f(x)} = 0$$
        en $R[x]/(f(x))$. pero, por la parte previa, ni $\ol{a(x)}$ ni $\ol{b(x)}$ son cero, por lo que $a(x)$ es un divisor de cero en $R[x]/(f(x))$.

        \item[d)] Supongamos que $a^m = 0 \in R$, Ahora $\ol{x}^{nm} = \ol{{(x^n)}^m} = \ol{a^m} = 0$, así que $\ol{x}$ es nilpotente en $R[x]/(f(x))$.
        
        \item[e)] Usando el pequeño teorema de Fermat, tenemos 
        $$ \ol{x-a}^p = \ol{(x-a)^p} = \ol{x^p-a^p} = \ol{a-a} = 0 $$
        Por lo que $\ol{x-a}$ es nilpotente en $R[x]/(f(x))$
    \end{enumerate}
\end{proof}