\begin{problem}[33]
\end{problem}

\begin{proof}
    \begin{enumerate}
        \item[a)] Sea $M \subseteq R$ un ideal maximal, y supongamos $M \neq M_c$ para todo $c \in [0,1]$. Entonces para todo $c$, existe una función $f_c \in M$ tal que $f_c(c) \neq 0$. Sin perdida de generalidad, asumamos que $f_c(c) > 0$. Como $f_c$ es continua, existe un número real positivo $\varepsilon_C > 0$ tal que $f_c[(c-\varepsilon_c, c+\varepsilon_c)] = 0$. Claramente, el conjunto 
        $$ \left\{ (c-\varepsilon_c, c+\varepsilon_c) \cap [0,1] \, | \, c \in [0,1] \right\}$$
        cubre $[0,1]$, ya que $[0,1]$ es compacto, esta cubierta tiene una subcubierta finita;  $K \subseteq [0,1]$ es finito y 
        $$ \left\{ (c-\varepsilon_c, c+\varepsilon_c) \cap [0,1] \, | \, c \in K \right\} $$
        cubre $[0,1]$. Ahora, para cada $c \in K$, definimos $u_c(x) = 1 + (x-c)/\varepsilon_c$ si $x \in (c-\varepsilon_c,c) \cap [0,1]$, $1 + (x-c)/\varepsilon_c$ si $x \in [c, c+\varepsilon_c] \cap [0,1]$, y cero en otros caso.

        Evidentemente $u_c \in R$ y $u_c$ desaparece fuera de $(c-\varepsilon_c, c+\varepsilon_x)$. Consideremos $g = \sum u_c f_c$, como $f_c \in M$, tenemos que $g \in M$. Sin embargo, para todo $x \in [0,1]$, $g(x)$ es positiva ya que cada $u_c(x)f_c(x)$ es no negativo y algunos $u_c(x)f_c(x)$ son positivos. Por lo que $g(x) > 0$ para todo $x$ y por tanto $1/g \in R$. Pero entonces $M$ contiene una unidad, lo cual es una contradicción, por lo que $M = M_c$ para algún $c \in [0,1]$.

        \item[b)] Supongamos que $b \neq c$, notemos que $x-b \in M_b$ pero $(x-b)(c) = c-b \neq 0$ así que $x-b \notin M_c$, por lo que $M_b \neq M_c$.
        
        \item[c)] Supongamos que $M_c = (x-c)$. Entonces, en particular, $\abs{c-x} = f(x)(x-c)$ para algún $f(x) \in R$. Para $x \neq c$ tenemos que $f(c) = \frac{\abs{x-c}}{x-c}$. Notemos que como $x$ se acerca a $c$ por la derecha, $f(x)$ se acerca a $-\infty$, mientras que $f(x)$ se acerca a $+\infty$ cuando $x$ se acerca a $c$ por la izquierda. En particular, ninguna extensión de $f$ está en $R$. Por lo que $M_c$ no es generado por $x-c$.
    
        \item[d)] supongamos que $M_c = (A)$ es finitamente generado, con $A = \left\{ a_i(x) | 1 \leq i \leq n \right\}$. Sea $f = \sum \abs{a_i}$; entonces $\sqrt{f}$ es continua en $[0,1]$. Más aún, tenemos que $\sqrt{f} \in M_c$. por lo que $\sqrt{f} = \sum r_ia_i$ para alguna función continua $r_i \in R$. Si dejamos $r = \sum \abs{r_i}$, tenemos que
        $$ \sqrt{f}(x) = \sum r_i(x)a_i(x) \leq \sum |r_i(x)||a_i(x)| \leq r(x)f(x) $$
        Notemos que para cada $b \neq c$, debe existir una función $a_i$ tal que $a_i(b) \neq 0$, Como en otros casos tenemos que $h(b) = 0$ para todo $h \in M_c$, y en particular $x-c \in M_c$. Por lo que $c$ es el único cero de $f$. como $\sqrt{f(x)} \leq r(x)f(x)$ para $x \neq c$, tenemos que $r(x) \geq 1 / \sqrt{f(x)}$. Si $x$ se acerca a $c$, $f(x)$ se aproxima a $0$, por lo que $1 / \sqrt{f(x)}$ no es acotado. Pero entonces $r(c)$ no existe, una contradicción, ya que $r(x) \in R$. Por lo que $M_c$ no es finitamente generado.
    \end{enumerate}
\end{proof}